% %%%%%%%%%%%%%%%%%%%%%%%%%%%%%%%%%%%%%%%%%%%%%%%%%%%%%%%%%%%%%%%%%%%%%%%%%%%%%
% 在这里填入题目
% %%%%%%%%%%%%%%%%%%%%%%%%%%%%%%%%%%%%%%%%%%%%%%%%%%%%%%%%%%%%%%%%%%%%%%%%%%%%%
\def\sectionName{简单博弈}



% 如果它是 beamer
\if 1\isBeamerMode\relax
    \section[\TOCName]{\sectionName}
\fi
% 如果它是 paper
\if 0\isBeamerMode\relax
    \section[\TOCName\ -\ \sectionName]{\sectionName}
\fi

\begin{frame}

% 如果它是 beamer
\if 1\isBeamerMode\relax
    {\Huge \sectionName}\par
\fi

% %%%%%%%%%%%%%%%%%%%%%%%%%%%%%%%%%%%%%%%%%%%%%%%%%%%%%%%%%%%%%%%%%%%%%%%%%%%%%
% 在这里填入你的名字
% %%%%%%%%%%%%%%%%%%%%%%%%%%%%%%%%%%%%%%%%%%%%%%%%%%%%%%%%%%%%%%%%%%%%%%%%%%%%%
\sectionAuthor{TheNameless}{TheNameless}



% %%%%%%%%%%%%%%%%%%%%%%%%%%%%%%%%%%%%%%%%%%%%%%%%%%%%%%%%%%%%%%%%%%%%%%%%%%%%%
% 这里可以写感想(嘲讽,bushi),也可以不写!!!
% %%%%%%%%%%%%%%%%%%%%%%%%%%%%%%%%%%%%%%%%%%%%%%%%%%%%%%%%%%%%%%%%%%%%%%%%%%%%%
博弈题。



\end{frame}

% %%%%%%%%%%%%%%%%%%%%%%%%%%%%%%%%%%%%%%%%%%%%%%%%%%%%%%%%%%%%%%%%%%%%%%%%%%%%%
% 这里开始写简单的题目意思 ~
% %%%%%%%%%%%%%%%%%%%%%%%%%%%%%%%%%%%%%%%%%%%%%%%%%%%%%%%%%%%%%%%%%%%%%%%%%%%%%
\subsection{题目意思}
\begin{frame} % 如果一个 frame 写不下的话,多开几个就好了~
Aob 和 Bob 进行游戏,其中 Alice 先手。

他们对同一个数字 $n$ 进行操作,在每一个属于自己的回合中,他们可以选择一个数字 $k
\in [l, r]$,并令 $n$ 自身减去 $k$。在自己的回合内,$n$ 被减为负数时,则失败。

双方均采取最优策略。问获胜者。
\end{frame}



% %%%%%%%%%%%%%%%%%%%%%%%%%%%%%%%%%%%%%%%%%%%%%%%%%%%%%%%%%%%%%%%%%%%%%%%%%%%%%
% 这里开始写题解 ~
% %%%%%%%%%%%%%%%%%%%%%%%%%%%%%%%%%%%%%%%%%%%%%%%%%%%%%%%%%%%%%%%%%%%%%%%%%%%%%
\subsection{题解}
\begin{frame} % 如果一个 frame 写不下的话,多开几个就好了~
从游戏规则中,我们可以发现,无论先手选的数字 $k$ 是多少,后手都可以选数字 $l + r
- k$,使得 $n$ 相比两人操作前减少 $l + r$。发现这个规律后,就不难找到必胜的办法
。

假设 $n \bmod (l + r) < l$  ($\bmod$为取余数操作) ,那么在 Alice 先手选数字 $k$
后,Bob只需要选数字 $l + r - k$,这样轮流操作多次后一定会在 Bob 操作完后使得原始
$n$ 变为 $n \bmod (l + r)$ ,此时无论 Alice 选什么数都会输。Bob 必胜。
\end{frame}


\begin{frame}
假设 $l \leq n \bmod (l + r) \leq r$ ,那么一开始的时候 Alice 可以选择 $n \bmod
(l + r)$。这样操作之后的新的 $n$ 就变为了 $(l + r)$ 的整数倍。在之后的操作中,无
论 Bob 取什么 $k$ ,Alice 都可以取 $l + r - k$。多次操作后,$n$ 一定会在Alice 操
作后变为 $0$,Bob 必输,Alice必赢。

假设 $n \bmod (l + r) > r$ ,那么一开始的时候 Alice 可以选择 $r$。这样操作之后的
新的 $n$ 一定满足 $n \bmod (l + r) < l$, 局面就变成了上面讨论后的情况,只是
Alice 和 Bob 的先后手对调了。因此,Alice 必赢。

综上,只需判断初始时 $n \bmod (l + r)$ 与 $l$ 大小关系就可以知道获胜者是谁了。
\end{frame}

