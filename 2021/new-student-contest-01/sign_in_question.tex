% %%%%%%%%%%%%%%%%%%%%%%%%%%%%%%%%%%%%%%%%%%%%%%%%%%%%%%%%%%%%%%%%%%%%%%%%%%%%%
% 在这里填入题目
% %%%%%%%%%%%%%%%%%%%%%%%%%%%%%%%%%%%%%%%%%%%%%%%%%%%%%%%%%%%%%%%%%%%%%%%%%%%%%
\def\sectionName{签到题?!}
\section{\sectionName}



\begin{frame}

% 如果它是 beamer
\if 1\isBeamerMode\relax
    {\Huge \sectionName}\par
\fi

% %%%%%%%%%%%%%%%%%%%%%%%%%%%%%%%%%%%%%%%%%%%%%%%%%%%%%%%%%%%%%%%%%%%%%%%%%%%%%
% 在这里填入你的名字
% %%%%%%%%%%%%%%%%%%%%%%%%%%%%%%%%%%%%%%%%%%%%%%%%%%%%%%%%%%%%%%%%%%%%%%%%%%%%%
\sectionAuthor{dyyyyyyyy}



% %%%%%%%%%%%%%%%%%%%%%%%%%%%%%%%%%%%%%%%%%%%%%%%%%%%%%%%%%%%%%%%%%%%%%%%%%%%%%
% 这里可以写感想(嘲讽,bushi),也可以不写!!!
% %%%%%%%%%%%%%%%%%%%%%%%%%%%%%%%%%%%%%%%%%%%%%%%%%%%%%%%%%%%%%%%%%%%%%%%%%%%%%
动态规划

或许某些 dalao 们一定会想到或许是什么高级的字符串数据结构

nonono,这是一道标程 30 行不到的动态规划题

如果把子序列改成子串的话那确实是后缀自动化上 $dp$ 的题



\end{frame}

% %%%%%%%%%%%%%%%%%%%%%%%%%%%%%%%%%%%%%%%%%%%%%%%%%%%%%%%%%%%%%%%%%%%%%%%%%%%%%
% 这里开始写简单的题目意思 ~
% %%%%%%%%%%%%%%%%%%%%%%%%%%%%%%%%%%%%%%%%%%%%%%%%%%%%%%%%%%%%%%%%%%%%%%%%%%%%%
\subsection{题目意思}
\begin{frame} % 如果一个 frame 写不下的话,多开几个就好了~
题目意思题目意思题目意思。
\end{frame}



% %%%%%%%%%%%%%%%%%%%%%%%%%%%%%%%%%%%%%%%%%%%%%%%%%%%%%%%%%%%%%%%%%%%%%%%%%%%%%
% 这里开始写题解 ~
% %%%%%%%%%%%%%%%%%%%%%%%%%%%%%%%%%%%%%%%%%%%%%%%%%%%%%%%%%%%%%%%%%%%%%%%%%%%%%
\subsection{题解}
\begin{frame} % 如果一个 frame 写不下的话,多开几个就好了~
我们先来看一个子问题:如何计算有多少个不同的子序列

\begin{itemize}
    \item dp[i] 表示 $s[i]$ 距离上次 $S[i]$ 这个数字出现这段距离内多出现的以 $S[i]$ 结尾的新子序列的个数
    \item 他的定义很复杂,但是转移感性理解起来是十分简单的
    \item 记 $last[i]$ 为 $s[i]$ 的上一次出现位置,如果没出现即为 $0$
    \item 对于 $last[i] + 1, ..., i - 1$ 的所有位置对应的序列都是对当前的 $dp[i]$ 有贡献的,不难想对于之前一样的数字,那么已经统计过了,就不需要统计了
\end{itemize}
\end{frame}


\begin{frame}[fragile]
直接看代码可能会更清楚一些

\begin{lstlisting}[style=C++]
dp[0] = 1;
for (int i = 1; i <= n; i++) {
    for (int j = i - 1; j >= 0; j--) {
        dp[i] += dp[j]; dp[i] %= mod;
        if (s[j] == s[i]) break;
    }
}
\end{lstlisting}
那么怎么处理带 $6$ 的字符串呢?先预处理掉或者遇到 $6$ 就 \texttt{continue}
\end{frame}


\begin{frame}
那么怎么求解数字和呢?这其实很好实现,另外维护一个数组 $ans$,和 $dp$ 类似,转移为
$$
ans[i]\leftarrow 10\times ans[j] + s[i]\times dp[j]
$$

时间复杂度:$O(|S||\sum|)$
\end{frame}


\begin{frame}[fragile]
主体代码如下:
\begin{lstlisting}[style=C++]
dp[0] = 1;
for (int i = 1; i <= n; i++) {
    if (s[i] == '6') continue;
    for (int j = i - 1; j >= 0; j--) {
        if (s[j] == '6') continue;
        dp[i] += dp[j]; dp[i] %= mod;
        ans[i] += (1ll * 10 * ans[j] + 1ll * (s[i] - '0') * dp[j]) % mod; ans[i] %= mod;
        if (s[j] == s[i]) break;
    }
    res += ans[i]; res %= mod;
}
\end{lstlisting}
\end{frame}
