% %%%%%%%%%%%%%%%%%%%%%%%%%%%%%%%%%%%%%%%%%%%%%%%%%%%%%%%%%%%%%%%%%%%%%%%%%%%%%
% 在这里填入题目
% %%%%%%%%%%%%%%%%%%%%%%%%%%%%%%%%%%%%%%%%%%%%%%%%%%%%%%%%%%%%%%%%%%%%%%%%%%%%%
\def\sectionName{xygg 的加乘法}



% 如果它是 beamer
\if 1\isBeamerMode\relax
    \section[\TOCName]{\sectionName}
\fi
% 如果它是 paper
\if 0\isBeamerMode\relax
    \section[\TOCName\ -\ \sectionName]{\sectionName}
\fi

\begin{frame}

% 如果它是 beamer
\if 1\isBeamerMode\relax
    {\Huge \sectionName}\par
\fi

% %%%%%%%%%%%%%%%%%%%%%%%%%%%%%%%%%%%%%%%%%%%%%%%%%%%%%%%%%%%%%%%%%%%%%%%%%%%%%
% 在这里填入你的名字
% %%%%%%%%%%%%%%%%%%%%%%%%%%%%%%%%%%%%%%%%%%%%%%%%%%%%%%%%%%%%%%%%%%%%%%%%%%%%%
\sectionAuthor{Emma194}{PeterlitsZo}



% %%%%%%%%%%%%%%%%%%%%%%%%%%%%%%%%%%%%%%%%%%%%%%%%%%%%%%%%%%%%%%%%%%%%%%%%%%%%%
% 这里可以写感想(嘲讽,bushi),也可以不写!!!
% %%%%%%%%%%%%%%%%%%%%%%%%%%%%%%%%%%%%%%%%%%%%%%%%%%%%%%%%%%%%%%%%%%%%%%%%%%%%%
什么牛马题?



\end{frame}

% %%%%%%%%%%%%%%%%%%%%%%%%%%%%%%%%%%%%%%%%%%%%%%%%%%%%%%%%%%%%%%%%%%%%%%%%%%%%%
% 这里开始写简单的题目意思 ~
% %%%%%%%%%%%%%%%%%%%%%%%%%%%%%%%%%%%%%%%%%%%%%%%%%%%%%%%%%%%%%%%%%%%%%%%%%%%%%
\subsection{题目意思}
\begin{frame} % 如果一个 frame 写不下的话,多开几个就好了~
给定 $n \le 10$ 个操作,和一个初始值 $x$。我们可以任意调换操作执行的顺序。

求模 $1 \times 10 ^ 9 + 7$ 之后最小的值。
\end{frame}



% %%%%%%%%%%%%%%%%%%%%%%%%%%%%%%%%%%%%%%%%%%%%%%%%%%%%%%%%%%%%%%%%%%%%%%%%%%%%%
% 这里开始写题解 ~
% %%%%%%%%%%%%%%%%%%%%%%%%%%%%%%%%%%%%%%%%%%%%%%%%%%%%%%%%%%%%%%%%%%%%%%%%%%%%%
\subsection{题解}
\begin{frame}[fragile]
首先我们不妨找出来一个朴素算法,即枚举所有的排列情况,然后选取最小的即可:
\begin{lstlisting}[style=C++]
int result = 0x3f3f3f3f;
for(; flg; flg = next_permutation(OP, OP + n)) {
    ll sub_res = x;
    for(int i = 0; i < n; i++) {
        if(OP[i].first == '+') {
            sub_res = (sub_res + OP[i].second) % MOD;
        } else {
            sub_res = (sub_res * OP[i].second) % MOD;
        }
    }
    result = min(result, (int)sub_res);
}
\end{lstlisting}
\end{frame}


\begin{frame} % 如果一个 frame 写不下的话,多开几个就好了~
但是所有的排列数为 $10! \simeq 3 \times 10^6$。而 $T = 1 \times 10 ^ 3$,仅仅只
有两个相乘便超过了 $5 \times 10^8$,故肯定会超时。

我们需要寻找一个优化的方法。
\end{frame}


\begin{frame}
我们可以发现,所有的操作均是一段乘法和一段加法。而加法乘法无论怎么搞,其次序都是
不影响结果的。我们可以把这个给剪掉,以降低复杂度。

那么最坏的情况不过是五个加法和五个乘法而已。而它的复杂度能证明出来是小于 $1
\times 10^6$,那么只要常数较小,就可以求解了。
\end{frame}

\begin{frame}
为了让常数较小,我们还需要预处理一下,即如果我们选择一些操作,我们必须得在平均复
杂度为 $O(1)$ 的时间内找出来这些操作的和或者积。
\end{frame}


\begin{frame}[fragile]
\begin{lstlisting}[style=C++]
int get_res(int i, int cur_op) {
    if(ANS[i][cur_op] != 0x3f3f3f3f)
        return ANS[i][cur_op];
    // 乘法的话,单位元是 1,不然是 0。
    int get = (cur_op == 1 ? 1 : 0);
    for(int j = 0; j < 12; j++) {
        if((i >> j) & 1) {
            if (cur_op == 1) get = (get * 1ll * OP[j].second) %MOD;
            else get = (get + OP[j].second) % MOD;
        }
    }
    return ANS[i][cur_op] = get;
}
\end{lstlisting}
\end{frame}


\begin{frame}
之后基于这些进行 DFS 搜索即可。搜索到叶子节点的时候进行更新。
\end{frame}
