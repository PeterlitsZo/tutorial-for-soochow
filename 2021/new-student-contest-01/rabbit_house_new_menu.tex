% %%%%%%%%%%%%%%%%%%%%%%%%%%%%%%%%%%%%%%%%%%%%%%%%%%%%%%%%%%%%%%%%%%%%%%%%%%%%%
% 在这里填入题目
% %%%%%%%%%%%%%%%%%%%%%%%%%%%%%%%%%%%%%%%%%%%%%%%%%%%%%%%%%%%%%%%%%%%%%%%%%%%%%
\def\sectionName{Rabbit House的新菜单}
\section[\TOCName]{\sectionName}



\begin{frame}

% 如果它是 beamer
\if 1\isBeamerMode\relax
    {\Huge \sectionName}\par
\fi

% %%%%%%%%%%%%%%%%%%%%%%%%%%%%%%%%%%%%%%%%%%%%%%%%%%%%%%%%%%%%%%%%%%%%%%%%%%%%%
% 在这里填入你的名字
% %%%%%%%%%%%%%%%%%%%%%%%%%%%%%%%%%%%%%%%%%%%%%%%%%%%%%%%%%%%%%%%%%%%%%%%%%%%%%
\sectionAuthor{lzc2001, KryptonAu, ddlearn}{dyyyyyyyy}



% %%%%%%%%%%%%%%%%%%%%%%%%%%%%%%%%%%%%%%%%%%%%%%%%%%%%%%%%%%%%%%%%%%%%%%%%%%%%%
% 这里可以写感想(嘲讽,bushi),也可以不写!!!
% %%%%%%%%%%%%%%%%%%%%%%%%%%%%%%%%%%%%%%%%%%%%%%%%%%%%%%%%%%%%%%%%%%%%%%%%%%%%%
经典且比较裸的容斥原理题。



\end{frame}

% %%%%%%%%%%%%%%%%%%%%%%%%%%%%%%%%%%%%%%%%%%%%%%%%%%%%%%%%%%%%%%%%%%%%%%%%%%%%%
% 这里开始写简单的题目意思 ~
% %%%%%%%%%%%%%%%%%%%%%%%%%%%%%%%%%%%%%%%%%%%%%%%%%%%%%%%%%%%%%%%%%%%%%%%%%%%%%
\subsection{题目意思}
\begin{frame} % 如果一个 frame 写不下的话,多开几个就好了~
给定 $n$ 个菜品,其中第 $i$ 个有两个属性值 $a_i$ 和 $b_i$。

我们需要计算满足下列条件的组合数:
\begin{enumerate}
    \item 选择 $3$ 个不同的菜品。
    \item 菜品的 $a_i$ 两两不相同或者 $b_i$ 两两不相同。
\end{enumerate}
\end{frame}



% %%%%%%%%%%%%%%%%%%%%%%%%%%%%%%%%%%%%%%%%%%%%%%%%%%%%%%%%%%%%%%%%%%%%%%%%%%%%%
% 这里开始写题解 ~
% %%%%%%%%%%%%%%%%%%%%%%%%%%%%%%%%%%%%%%%%%%%%%%%%%%%%%%%%%%%%%%%%%%%%%%%%%%%%%
\subsection{题解}


\begin{frame} % 如果一个 frame 写不下的话,多开几个就好了~
我们需要记录三个东西:

\begin{enumerate}
    \item \cmd{map<pair<int, int>, int> cnt}:\cmd{cnt[\{x, y\}]} 表示点对 $(x,
        y)$ 的菜品有多少个
    \item \cmd{map<int, int> cnta}:\cmd{cnta[x]} 表示 $a_i = x$ 的 菜品有多少个
    \item \cmd{map<int, int> cntb}:\cmd{cntb[x]} 表示 $b_i = x$ 的 菜品有多少个
\end{enumerate}
\end{frame}


\begin{frame} % 如果一个 frame 写不下的话,多开几个就好了~

如果不考虑约束条件,所有的选择方法为 ${n \choose 3}$。

先在我们考虑计算不满足约束条件的情况 $ans_v$。

有哪些情况不满足呢?

对于每一种 $(a_i, b_i)$ 来说,
\begin{itemize}
    \item 为方便表示,令:$v =\ $\cmd{cnt[\{$a_i$, $b_i$\}]},$x =\ $
        \cmd{cnta[$a_i$]},$y =\ $\cmd{cntb[$b_i$]}。
	\item 三个菜品的美味度和观赏度都一样,有 ${v \choose 3}$ 种。
	\item 两个菜品的美味度和观赏度都一样,有 ${v \choose 2} \times (n - v)$ 种。
	\item 两个菜品的美味度一样,两个菜品的观赏度一样,有 ${v\times (x - v)\times
        (y - v)}$ 种。
\end{itemize}

用总情况减去不满足约束的情况就是满足约束的情况。

时间复杂度:$O(N\log N)$。
\end{frame}


