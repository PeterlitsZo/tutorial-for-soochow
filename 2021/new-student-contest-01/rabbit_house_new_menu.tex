% %%%%%%%%%%%%%%%%%%%%%%%%%%%%%%%%%%%%%%%%%%%%%%%%%%%%%%%%%%%%%%%%%%%%%%%%%%%%%
% 在这里填入题目
% %%%%%%%%%%%%%%%%%%%%%%%%%%%%%%%%%%%%%%%%%%%%%%%%%%%%%%%%%%%%%%%%%%%%%%%%%%%%%
\def\sectionName{Rabbit House的新菜单}
\section{\sectionName}



\begin{frame}

% 如果它是 beamer
\if 1\isBeamerMode\relax
    {\Huge \sectionName}\par
\fi

% %%%%%%%%%%%%%%%%%%%%%%%%%%%%%%%%%%%%%%%%%%%%%%%%%%%%%%%%%%%%%%%%%%%%%%%%%%%%%
% 在这里填入你的名字
% %%%%%%%%%%%%%%%%%%%%%%%%%%%%%%%%%%%%%%%%%%%%%%%%%%%%%%%%%%%%%%%%%%%%%%%%%%%%%
\sectionAuthor{dyyyyyyyy}



% %%%%%%%%%%%%%%%%%%%%%%%%%%%%%%%%%%%%%%%%%%%%%%%%%%%%%%%%%%%%%%%%%%%%%%%%%%%%%
% 这里可以写感想(嘲讽,bushi),也可以不写!!!
% %%%%%%%%%%%%%%%%%%%%%%%%%%%%%%%%%%%%%%%%%%%%%%%%%%%%%%%%%%%%%%%%%%%%%%%%%%%%%
经典且比较裸的容斥原理题



\end{frame}

% %%%%%%%%%%%%%%%%%%%%%%%%%%%%%%%%%%%%%%%%%%%%%%%%%%%%%%%%%%%%%%%%%%%%%%%%%%%%%
% 这里开始写简单的题目意思 ~
% %%%%%%%%%%%%%%%%%%%%%%%%%%%%%%%%%%%%%%%%%%%%%%%%%%%%%%%%%%%%%%%%%%%%%%%%%%%%%
\subsection{题目意思}
\begin{frame} % 如果一个 frame 写不下的话,多开几个就好了~
题目意思题目意思题目意思。
\end{frame}



% %%%%%%%%%%%%%%%%%%%%%%%%%%%%%%%%%%%%%%%%%%%%%%%%%%%%%%%%%%%%%%%%%%%%%%%%%%%%%
% 这里开始写题解 ~
% %%%%%%%%%%%%%%%%%%%%%%%%%%%%%%%%%%%%%%%%%%%%%%%%%%%%%%%%%%%%%%%%%%%%%%%%%%%%%
\subsection{题解}


\begin{frame} % 如果一个 frame 写不下的话,多开几个就好了~
我们需要记录三个东西

\begin{enumerate}
	\item map<pair<int, int>, int> cnt:cnt[(x, y)] 表示点对 $(x, y)$ 的菜品有多少个
	\item map<int, int> cnta:cnta[x] 表示 $a_i = x$ 的 菜品有多少个
	\item map<int, int> cntb:cntb[x] 表示 $b_i = x$ 的 菜品有多少个
\end{enumerate}
\end{frame}


\begin{frame} % 如果一个 frame 写不下的话,多开几个就好了~

如果不考虑约束条件,所有的选择方法为 ${n \choose 3}$

先在我们考虑计算不满足约束条件的情况 $ans_v$

有哪些情况不满足呢?

对于每一种 $(a_i, b_i)$ 来说,

\begin{itemize}
	\item 为方便表示,$cnt[(a_i, b_i)] = v, cnta[a_i] = x, cntb[b_i] = y$
	\item 三个菜品的美味度和观赏度都一样,有 ${v \choose 3}$ 种
	\item 两个菜品的美味度和观赏度都一样,有 ${v \choose 2} \times (n - v)$ 种
	\item 两个菜品的美味度一样,两个菜品的观赏度一样,有 ${v\times (x - v)\times (y - v)}$ 种
\end{itemize}

用总情况减去不满足约束的情况就是满足约束的情况

时间复杂度:$O(N\log N)$
\end{frame}


