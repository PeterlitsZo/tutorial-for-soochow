% %%%%%%%%%%%%%%%%%%%%%%%%%%%%%%%%%%%%%%%%%%%%%%%%%%%%%%%%%%%%%%%%%%%%%%%%%%%%%
% 选择 isBeamerMode -> 0: 文件,1:幻灯片
% %%%%%%%%%%%%%%%%%%%%%%%%%%%%%%%%%%%%%%%%%%%%%%%%%%%%%%%%%%%%%%%%%%%%%%%%%%%%%
\def\isBeamerMode{1}
\if 0\isBeamerMode
    \documentclass[lang=cn]{elegantpaper}
    \let\chapter\section
    \renewenvironment{frame}{}{}
\else
    \documentclass{ctexbeamer}
    \usepackage{listings}
    % 幻灯片的一些配置
    \usetheme{Berlin}
    \AtBeginSection[] {
        \begin{frame}
            \frametitle{目录}
            \tableofcontents[currentsection]
        \end{frame}
    }
\fi



% %%%%%%%%%%%%%%%%%%%%%%%%%%%%%%%%%%%%%%%%%%%%%%%%%%%%%%%%%%%%%%%%%%%%%%%%%%%%%
% 一些宏定义和相关配置
% %%%%%%%%%%%%%%%%%%%%%%%%%%%%%%%%%%%%%%%%%%%%%%%%%%%%%%%%%%%%%%%%%%%%%%%%%%%%%
% \sectionAuthor{<作者姓名>}
% 用于添加章节作者
\newcommand\sectionAuthor[1]{%
    \if 0\isBeamerMode\relax \vspace*{-0.3cm} \else \vspace*{0.3cm} \fi%
    \noindent By \textbf{#1}.%
    \if 0\isBeamerMode\relax \vspace*{0.2cm} \else \vspace*{0.5cm} \fi}

% \addCode{<不含后缀的 CPP 文件名>}
% 用于陈列代码
\newcommand\addCode[1]{%
    \subsection{#1}%
    \lstinputlisting[style=C++]{#1.cpp}}%

% \cmd{<代码>}
% 用于陈列行内代码
\newcommand\cmd[1]{%
    \texttt{#1}}

% 配置陈列代码的风格
\lstset{
    basicstyle          =   \sffamily,          % 基本代码风格
    keywordstyle        =   \bfseries,          % 关键字风格
    commentstyle        =   \rmfamily\itshape,  % 注释的风格,斜体
    stringstyle         =   \ttfamily,          % 字符串风格
    flexiblecolumns,                            % 别问为什么,加上这个
    numbers             =   left,               % 行号的位置在左边
    showspaces          =   false,              % 是否显示空格,显示了有点乱,所以不现实了
    numberstyle         =   \ttfamily,          % 行号的样式,小五号,tt等宽字体
    showstringspaces    =   false,
    captionpos          =   t,                  % 这段代码的名字所呈现的位置,t指的是top上面
    frame               =   lrtb,               % 显示边框
}

\lstdefinestyle{C++}{
    language        =   C++,
    basicstyle      =   \ttfamily,
    numberstyle     =   \ttfamily,
    keywordstyle    =   \color{blue},
    keywordstyle    =   [2] \color{teal},
    stringstyle     =   \color{magenta},
    commentstyle    =   \color{red}\ttfamily,
    breaklines      =   true,                   % 自动换行,建议不要写太长的行
    columns         =   fixed,                  % 如果不加这一句,字间距就不固定,很丑,必须加
    basewidth       =   0.5em,
    tabsize         =   4,
}



% %%%%%%%%%%%%%%%%%%%%%%%%%%%%%%%%%%%%%%%%%%%%%%%%%%%%%%%%%%%%%%%%%%%%%%%%%%%%%
% 在这里编辑基础信息
% %%%%%%%%%%%%%%%%%%%%%%%%%%%%%%%%%%%%%%%%%%%%%%%%%%%%%%%%%%%%%%%%%%%%%%%%%%%%%
\title{新生赛题解}
\author{苏州大学 ACM 集训队}



\begin{document}

% %%%%%%%%%%%%%%%%%%%%%%%%%%%%%%%%%%%%%%%%%%%%%%%%%%%%%%%%%%%%%%%%%%%%%%%%%%%%%
% 生成标题
% %%%%%%%%%%%%%%%%%%%%%%%%%%%%%%%%%%%%%%%%%%%%%%%%%%%%%%%%%%%%%%%%%%%%%%%%%%%%%
% 如果它是 paper
\if 0\isBeamerMode\relax
    \maketitle
    \tableofcontents
\fi

% 如果它是 beamer
\if 1\isBeamerMode\relax
    \frame{\titlepage}
\fi



% %%%%%%%%%%%%%%%%%%%%%%%%%%%%%%%%%%%%%%%%%%%%%%%%%%%%%%%%%%%%%%%%%%%%%%%%%%%%%
% 在这里 include 所有题解
% %%%%%%%%%%%%%%%%%%%%%%%%%%%%%%%%%%%%%%%%%%%%%%%%%%%%%%%%%%%%%%%%%%%%%%%%%%%%%
% %%%%%%%%%%%%%%%%%%%%%%%%%%%%%%%%%%%%%%%%%%%%%%%%%%%%%%%%%%%%%%%%%%%%%%%%%%%%%
% 在这里填入题目
% %%%%%%%%%%%%%%%%%%%%%%%%%%%%%%%%%%%%%%%%%%%%%%%%%%%%%%%%%%%%%%%%%%%%%%%%%%%%%
\def\sectionName{Rabbit House的新菜单}



% 如果它是 beamer
\if 1\isBeamerMode\relax
    \section[\TOCName]{\sectionName}
\fi
% 如果它是 paper
\if 0\isBeamerMode\relax
    \section[\TOCName\ -\ \sectionName]{\sectionName}
\fi

\begin{frame}

% 如果它是 beamer
\if 1\isBeamerMode\relax
    {\Huge \sectionName}\par
\fi

% %%%%%%%%%%%%%%%%%%%%%%%%%%%%%%%%%%%%%%%%%%%%%%%%%%%%%%%%%%%%%%%%%%%%%%%%%%%%%
% 在这里填入你的名字
% %%%%%%%%%%%%%%%%%%%%%%%%%%%%%%%%%%%%%%%%%%%%%%%%%%%%%%%%%%%%%%%%%%%%%%%%%%%%%
\sectionAuthor{lzc2001, KryptonAu, ddlearn}{dyyyyyyyy}



% %%%%%%%%%%%%%%%%%%%%%%%%%%%%%%%%%%%%%%%%%%%%%%%%%%%%%%%%%%%%%%%%%%%%%%%%%%%%%
% 这里可以写感想(嘲讽,bushi),也可以不写!!!
% %%%%%%%%%%%%%%%%%%%%%%%%%%%%%%%%%%%%%%%%%%%%%%%%%%%%%%%%%%%%%%%%%%%%%%%%%%%%%
经典且比较裸的容斥原理题。



\end{frame}

% %%%%%%%%%%%%%%%%%%%%%%%%%%%%%%%%%%%%%%%%%%%%%%%%%%%%%%%%%%%%%%%%%%%%%%%%%%%%%
% 这里开始写简单的题目意思 ~
% %%%%%%%%%%%%%%%%%%%%%%%%%%%%%%%%%%%%%%%%%%%%%%%%%%%%%%%%%%%%%%%%%%%%%%%%%%%%%
\subsection{题目意思}
\begin{frame} % 如果一个 frame 写不下的话,多开几个就好了~
给定 $n$ 个菜品,其中第 $i$ 个有两个属性值 $a_i$ 和 $b_i$。

我们需要计算满足下列条件的组合数:
\begin{enumerate}
    \item 选择 $3$ 个不同的菜品。
    \item 菜品的 $a_i$ 两两不相同或者 $b_i$ 两两不相同。
\end{enumerate}
\end{frame}



% %%%%%%%%%%%%%%%%%%%%%%%%%%%%%%%%%%%%%%%%%%%%%%%%%%%%%%%%%%%%%%%%%%%%%%%%%%%%%
% 这里开始写题解 ~
% %%%%%%%%%%%%%%%%%%%%%%%%%%%%%%%%%%%%%%%%%%%%%%%%%%%%%%%%%%%%%%%%%%%%%%%%%%%%%
\subsection{题解}


\begin{frame} % 如果一个 frame 写不下的话,多开几个就好了~
我们需要记录三个东西:

\begin{enumerate}
    \item \cmd{map<pair<int, int>, int> cnt}:\cmd{cnt[\{x, y\}]} 表示点对 $(x,
        y)$ 的菜品有多少个
    \item \cmd{map<int, int> cnta}:\cmd{cnta[x]} 表示 $a_i = x$ 的 菜品有多少个
    \item \cmd{map<int, int> cntb}:\cmd{cntb[x]} 表示 $b_i = x$ 的 菜品有多少个
\end{enumerate}
\end{frame}


\begin{frame} % 如果一个 frame 写不下的话,多开几个就好了~

如果不考虑约束条件,所有的选择方法为 ${n \choose 3}$。

先在我们考虑计算不满足约束条件的情况 $ans_v$。

有哪些情况不满足呢?

对于每一种 $(a_i, b_i)$ 来说,
\begin{itemize}
    \item 为方便表示,令:$v =\ $\cmd{cnt[\{$a_i$, $b_i$\}]},$x =\ $
        \cmd{cnta[$a_i$]},$y =\ $\cmd{cntb[$b_i$]}。
	\item 三个菜品的美味度和观赏度都一样,有 ${v \choose 3}$ 种。
	\item 两个菜品的美味度和观赏度都一样,有 ${v \choose 2} \times (n - v)$ 种。
	\item 两个菜品的美味度一样,两个菜品的观赏度一样,有 ${v\times (x - v)\times
        (y - v)}$ 种。
\end{itemize}

用总情况减去不满足约束的情况就是满足约束的情况。

时间复杂度:$O(N\log N)$。
\end{frame}



% %%%%%%%%%%%%%%%%%%%%%%%%%%%%%%%%%%%%%%%%%%%%%%%%%%%%%%%%%%%%%%%%%%%%%%%%%%%%%
% 在这里填入题目
% %%%%%%%%%%%%%%%%%%%%%%%%%%%%%%%%%%%%%%%%%%%%%%%%%%%%%%%%%%%%%%%%%%%%%%%%%%%%%
\def\sectionName{Peter 和他的王国 1}
\section[\TOCName]{\sectionName}



\begin{frame}

% 如果它是 beamer
\if 1\isBeamerMode\relax
    {\Huge \sectionName}\par
\fi

% %%%%%%%%%%%%%%%%%%%%%%%%%%%%%%%%%%%%%%%%%%%%%%%%%%%%%%%%%%%%%%%%%%%%%%%%%%%%%
% 在这里填入你的名字
% %%%%%%%%%%%%%%%%%%%%%%%%%%%%%%%%%%%%%%%%%%%%%%%%%%%%%%%%%%%%%%%%%%%%%%%%%%%%%
\sectionAuthor{Peterlits Zo}{Peterlits Zo}



% %%%%%%%%%%%%%%%%%%%%%%%%%%%%%%%%%%%%%%%%%%%%%%%%%%%%%%%%%%%%%%%%%%%%%%%%%%%%%
% 这里可以写感想(嘲讽,bushi),也可以不写!!!
% %%%%%%%%%%%%%%%%%%%%%%%%%%%%%%%%%%%%%%%%%%%%%%%%%%%%%%%%%%%%%%%%%%%%%%%%%%%%%
道理我都懂,不过为啥这个题目后面有一个数字 1 呢?



\end{frame}

% %%%%%%%%%%%%%%%%%%%%%%%%%%%%%%%%%%%%%%%%%%%%%%%%%%%%%%%%%%%%%%%%%%%%%%%%%%%%%
% 这里开始写简单的题目意思 ~
% %%%%%%%%%%%%%%%%%%%%%%%%%%%%%%%%%%%%%%%%%%%%%%%%%%%%%%%%%%%%%%%%%%%%%%%%%%%%%
\subsection{题目意思}
\begin{frame} % 如果一个 frame 写不下的话,多开几个就好了~
给定一个数组 $A$,第 $i$ 个元素为 $A_i$。

构造一个集合,即:\[\{x \mid A_i \in A, A_j \in A, x \mid A_i, x \mid A_j, x \le
l\}\]求该集合中的最大值。
\end{frame}



% %%%%%%%%%%%%%%%%%%%%%%%%%%%%%%%%%%%%%%%%%%%%%%%%%%%%%%%%%%%%%%%%%%%%%%%%%%%%%
% 这里开始写题解 ~
% %%%%%%%%%%%%%%%%%%%%%%%%%%%%%%%%%%%%%%%%%%%%%%%%%%%%%%%%%%%%%%%%%%%%%%%%%%%%%
\subsection{题解}
\begin{frame} % 如果一个 frame 写不下的话,多开几个就好了~
我们知道,最大公约数能够被其他所有的公约数整除,那么如果我们求出最大公因数的所有
因数,然后二分即可得到在 $[1, l]$ 区间内的公约数。时间复杂度为 $O(\sqrt P)$。其
中分解质因数的时间复杂度是 $O(\sqrt P)$,而二分查找是 $O(\log \sqrt P)$,可以忽
略不计。

总复杂度为两两组合乘以上面的复杂度,故为 $O(n^2 \sqrt P)$。
\end{frame}



% %%%%%%%%%%%%%%%%%%%%%%%%%%%%%%%%%%%%%%%%%%%%%%%%%%%%%%%%%%%%%%%%%%%%%%%%%%%%%
% 在这里 include 所有参考代码
% %%%%%%%%%%%%%%%%%%%%%%%%%%%%%%%%%%%%%%%%%%%%%%%%%%%%%%%%%%%%%%%%%%%%%%%%%%%%%
% 如果它是 paper
\if 0\isBeamerMode\relax
    \section{参考代码}
    \addCode{title}
\fi



\end{document}
