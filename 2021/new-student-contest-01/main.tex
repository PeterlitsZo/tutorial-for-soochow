% %%%%%%%%%%%%%%%%%%%%%%%%%%%%%%%%%%%%%%%%%%%%%%%%%%%%%%%%%%%%%%%%%%%%%%%%%%%%%
% 选择 isBeamerMode -> 0: 文件,1:幻灯片
% %%%%%%%%%%%%%%%%%%%%%%%%%%%%%%%%%%%%%%%%%%%%%%%%%%%%%%%%%%%%%%%%%%%%%%%%%%%%%
\def\isBeamerMode{1}
\if 0\isBeamerMode
    \documentclass[lang=cn]{elegantpaper}
    \let\chapter\section
    \renewenvironment{frame}{}{}
\else
    \documentclass{ctexbeamer}
    \usepackage{listings}
    % 幻灯片的一些配置
    \usetheme{Berlin}
    \AtBeginSection[] {
        \begin{frame}
            \frametitle{目录}
            \tableofcontents[currentsection, hideallsubsections]
        \end{frame}
    }
\fi



% %%%%%%%%%%%%%%%%%%%%%%%%%%%%%%%%%%%%%%%%%%%%%%%%%%%%%%%%%%%%%%%%%%%%%%%%%%%%%
% 一些宏定义和相关配置
% %%%%%%%%%%%%%%%%%%%%%%%%%%%%%%%%%%%%%%%%%%%%%%%%%%%%%%%%%%%%%%%%%%%%%%%%%%%%%
% \sectionAuthor{<作者姓名>}
% 用于添加章节作者
\newcommand\sectionAuthor[1]{%
    \if 0\isBeamerMode\relax \vspace*{-1cm}\hspace*{\stretch{5}}\dotfill \else \vspace*{0.3cm} \fi%
    By \textbf{#1}.%
    \if 0\isBeamerMode\relax \vspace*{0.3cm} \else \vspace*{0.5cm} \fi}

% \addCode{<不含后缀的 CPP 文件名>}
% 用于陈列代码
\newcommand\addCode[1]{%
    \subsection{#1}%
    \lstinputlisting[style=C++]{#1.cpp}}%

% \cmd{<代码>}
% 用于陈列行内代码
\newcommand\cmd[1]{%
    \texttt{#1}}

% 配置陈列代码的风格
\lstset{
    basicstyle          =   \sffamily,          % 基本代码风格
    keywordstyle        =   \bfseries,          % 关键字风格
    commentstyle        =   \rmfamily\itshape,  % 注释的风格,斜体
    stringstyle         =   \ttfamily,          % 字符串风格
    flexiblecolumns,                            % 别问为什么,加上这个
    numbers             =   left,               % 行号的位置在左边
    showspaces          =   false,              % 是否显示空格,显示了有点乱,所以不现实了
    numberstyle         =   \ttfamily,          % 行号的样式,小五号,tt等宽字体
    showstringspaces    =   false,
    captionpos          =   t,                  % 这段代码的名字所呈现的位置,t指的是top上面
    frame               =   lrtb,               % 显示边框
}

\lstdefinestyle{C++}{
    language        =   C++,
    basicstyle      =   \ttfamily,
    numberstyle     =   \ttfamily,
    keywordstyle    =   \color{blue},
    keywordstyle    =   [2] \color{teal},
    stringstyle     =   \color{magenta},
    commentstyle    =   \color{red}\ttfamily,
    breaklines      =   true,                   % 自动换行,建议不要写太长的行
    columns         =   fixed,                  % 如果不加这一句,字间距就不固定,很丑,必须加
    basewidth       =   0.5em,
    tabsize         =   4,
}



% %%%%%%%%%%%%%%%%%%%%%%%%%%%%%%%%%%%%%%%%%%%%%%%%%%%%%%%%%%%%%%%%%%%%%%%%%%%%%
% 在这里编辑基础信息
% %%%%%%%%%%%%%%%%%%%%%%%%%%%%%%%%%%%%%%%%%%%%%%%%%%%%%%%%%%%%%%%%%%%%%%%%%%%%%
\title{新生赛题解}
\author{苏州大学 ACM 集训队}



\begin{document}

% %%%%%%%%%%%%%%%%%%%%%%%%%%%%%%%%%%%%%%%%%%%%%%%%%%%%%%%%%%%%%%%%%%%%%%%%%%%%%
% 生成标题
% %%%%%%%%%%%%%%%%%%%%%%%%%%%%%%%%%%%%%%%%%%%%%%%%%%%%%%%%%%%%%%%%%%%%%%%%%%%%%
% 如果它是 paper
\if 0\isBeamerMode\relax
    \maketitle
    \tableofcontents
\fi

% 如果它是 beamer
\if 1\isBeamerMode\relax
    \frame{\titlepage}
\fi



% %%%%%%%%%%%%%%%%%%%%%%%%%%%%%%%%%%%%%%%%%%%%%%%%%%%%%%%%%%%%%%%%%%%%%%%%%%%%%
% 在这里 include 所有题解
% %%%%%%%%%%%%%%%%%%%%%%%%%%%%%%%%%%%%%%%%%%%%%%%%%%%%%%%%%%%%%%%%%%%%%%%%%%%%%
% %%%%%%%%%%%%%%%%%%%%%%%%%%%%%%%%%%%%%%%%%%%%%%%%%%%%%%%%%%%%%%%%%%%%%%%%%%%%%
% 在这里填入题目
% %%%%%%%%%%%%%%%%%%%%%%%%%%%%%%%%%%%%%%%%%%%%%%%%%%%%%%%%%%%%%%%%%%%%%%%%%%%%%
\def\sectionName{简单博弈}
\section[\TOCName]{\sectionName}



\begin{frame}

% 如果它是 beamer
\if 1\isBeamerMode\relax
    {\Huge \sectionName}\par
\fi

% %%%%%%%%%%%%%%%%%%%%%%%%%%%%%%%%%%%%%%%%%%%%%%%%%%%%%%%%%%%%%%%%%%%%%%%%%%%%%
% 在这里填入你的名字
% %%%%%%%%%%%%%%%%%%%%%%%%%%%%%%%%%%%%%%%%%%%%%%%%%%%%%%%%%%%%%%%%%%%%%%%%%%%%%
\sectionAuthor{wpz}



% %%%%%%%%%%%%%%%%%%%%%%%%%%%%%%%%%%%%%%%%%%%%%%%%%%%%%%%%%%%%%%%%%%%%%%%%%%%%%
% 这里可以写感想(嘲讽,bushi),也可以不写!!!
% %%%%%%%%%%%%%%%%%%%%%%%%%%%%%%%%%%%%%%%%%%%%%%%%%%%%%%%%%%%%%%%%%%%%%%%%%%%%%
博弈题。



\end{frame}

% %%%%%%%%%%%%%%%%%%%%%%%%%%%%%%%%%%%%%%%%%%%%%%%%%%%%%%%%%%%%%%%%%%%%%%%%%%%%%
% 这里开始写简单的题目意思 ~
% %%%%%%%%%%%%%%%%%%%%%%%%%%%%%%%%%%%%%%%%%%%%%%%%%%%%%%%%%%%%%%%%%%%%%%%%%%%%%
\subsection{题目意思}
\begin{frame} % 如果一个 frame 写不下的话,多开几个就好了~
Aob 和 Bob 进行游戏,其中 Alice 先手。

他们对同一个数字 $n$ 进行操作,在每一个属于自己的回合中,他们可以选择一个数字 $k
\in [l, r]$,并令 $n$ 自身减去 $k$。在自己的回合内,$n$ 被减为负数时,则失败。

双方均采取最优策略。问获胜者。
\end{frame}



% %%%%%%%%%%%%%%%%%%%%%%%%%%%%%%%%%%%%%%%%%%%%%%%%%%%%%%%%%%%%%%%%%%%%%%%%%%%%%
% 这里开始写题解 ~
% %%%%%%%%%%%%%%%%%%%%%%%%%%%%%%%%%%%%%%%%%%%%%%%%%%%%%%%%%%%%%%%%%%%%%%%%%%%%%
\subsection{题解}
\begin{frame} % 如果一个 frame 写不下的话,多开几个就好了~
从游戏规则中,我们可以发现,无论先手选的数字 $k$ 是多少,后手都可以选数字 $l + r
- k$,使得 $n$ 相比两人操作前减少 $l + r$。发现这个规律后,就不难找到必胜的办法
。

假设 $n \bmod (l + r) < l$  ($\bmod$为取余数操作) ,那么在 Alice 先手选数字 $k$
后,Bob只需要选数字 $l + r - k$,这样轮流操作多次后一定会在 Bob 操作完后使得原始
$n$ 变为 $n \bmod (l + r)$ ,此时无论 Alice 选什么数都会输。Bob 必胜。
\end{frame}


\begin{frame}
假设 $l \leq n \bmod (l + r) \leq r$ ,那么一开始的时候 Alice 可以选择 $n \bmod
(l + r)$。这样操作之后的新的 $n$ 就变为了 $(l + r)$ 的整数倍。在之后的操作中,无
论 Bob 取什么 $k$ ,Alice 都可以取 $l + r - k$。多次操作后,$n$ 一定会在Alice 操
作后变为 $0$,Bob 必输,Alice必赢。

假设 $n \bmod (l + r) > r$ ,那么一开始的时候 Alice 可以选择 $r$。这样操作之后的
新的 $n$ 一定满足 $n \bmod (l + r) < l$, 局面就变成了上面讨论后的情况,只是
Alice 和 Bob 的先后手对调了。因此,Alice 必赢。

综上,只需判断初始时 $n \bmod (l + r)$ 与 $l$ 大小关系就可以知道获胜者是谁了。
\end{frame}


% %%%%%%%%%%%%%%%%%%%%%%%%%%%%%%%%%%%%%%%%%%%%%%%%%%%%%%%%%%%%%%%%%%%%%%%%%%%%%
% 在这里填入题目
% %%%%%%%%%%%%%%%%%%%%%%%%%%%%%%%%%%%%%%%%%%%%%%%%%%%%%%%%%%%%%%%%%%%%%%%%%%%%%
\def\sectionName{签到题?!}
\section{\sectionName}



\begin{frame}

% 如果它是 beamer
\if 1\isBeamerMode\relax
    {\Huge \sectionName}\par
\fi

% %%%%%%%%%%%%%%%%%%%%%%%%%%%%%%%%%%%%%%%%%%%%%%%%%%%%%%%%%%%%%%%%%%%%%%%%%%%%%
% 在这里填入你的名字
% %%%%%%%%%%%%%%%%%%%%%%%%%%%%%%%%%%%%%%%%%%%%%%%%%%%%%%%%%%%%%%%%%%%%%%%%%%%%%
\sectionAuthor{dyyyyyyyy}



% %%%%%%%%%%%%%%%%%%%%%%%%%%%%%%%%%%%%%%%%%%%%%%%%%%%%%%%%%%%%%%%%%%%%%%%%%%%%%
% 这里可以写感想(嘲讽,bushi),也可以不写!!!
% %%%%%%%%%%%%%%%%%%%%%%%%%%%%%%%%%%%%%%%%%%%%%%%%%%%%%%%%%%%%%%%%%%%%%%%%%%%%%
动态规划

或许某些 dalao 们一定会想到或许是什么高级的字符串数据结构

nonono,这是一道标程 30 行不到的动态规划题

如果把子序列改成子串的话那确实是后缀自动化上 $dp$ 的题



\end{frame}

% %%%%%%%%%%%%%%%%%%%%%%%%%%%%%%%%%%%%%%%%%%%%%%%%%%%%%%%%%%%%%%%%%%%%%%%%%%%%%
% 这里开始写简单的题目意思 ~
% %%%%%%%%%%%%%%%%%%%%%%%%%%%%%%%%%%%%%%%%%%%%%%%%%%%%%%%%%%%%%%%%%%%%%%%%%%%%%
\subsection{题目意思}
\begin{frame} % 如果一个 frame 写不下的话,多开几个就好了~
题目意思题目意思题目意思。
\end{frame}



% %%%%%%%%%%%%%%%%%%%%%%%%%%%%%%%%%%%%%%%%%%%%%%%%%%%%%%%%%%%%%%%%%%%%%%%%%%%%%
% 这里开始写题解 ~
% %%%%%%%%%%%%%%%%%%%%%%%%%%%%%%%%%%%%%%%%%%%%%%%%%%%%%%%%%%%%%%%%%%%%%%%%%%%%%
\subsection{题解}
\begin{frame} % 如果一个 frame 写不下的话,多开几个就好了~
我们先来看一个子问题:如何计算有多少个不同的子序列

\begin{itemize}
    \item dp[i] 表示 $s[i]$ 距离上次 $S[i]$ 这个数字出现这段距离内多出现的以 $S[i]$ 结尾的新子序列的个数
    \item 他的定义很复杂,但是转移感性理解起来是十分简单的
    \item 记 $last[i]$ 为 $s[i]$ 的上一次出现位置,如果没出现即为 $0$
    \item 对于 $last[i] + 1, ..., i - 1$ 的所有位置对应的序列都是对当前的 $dp[i]$ 有贡献的,不难想对于之前一样的数字,那么已经统计过了,就不需要统计了
\end{itemize}
\end{frame}

\begin{frame}[fragile]

\begin{minipage}{0.8\textwidth}

直接看代码可能会更清楚一些

\begin{lstlisting}[style=C++]
dp[0] = 1;
for (int i = 1; i <= n; i++) {
    for (int j = i - 1; j >= 0; j--) {
        dp[i] += dp[j]; dp[i] %= mod;
        if (s[j] == s[i]) break;
    }
}
\end{lstlisting}
那么怎么处理带 $6$ 的字符串呢?先预处理掉或者遇到 $6$ 就 \texttt{continue}

\end{minipage}

\end{frame}

\begin{frame}
那么怎么求解数字和呢?这其实很好实现,另外维护一个数组 $ans$,和 $dp$ 类似,转移为
$$
ans[i]\leftarrow 10\times ans[j] + s[i]\times dp[j]
$$

时间复杂度:$O(|S||\sum|)$
\end{frame}

\begin{frame}[fragile]
\begin{minipage}{0.8\textwidth}

主体代码如下:
\begin{lstlisting}[style=C++]
dp[0] = 1;
for (int i = 1; i <= n; i++) {
    if (s[i] == '6') continue;
    for (int j = i - 1; j >= 0; j--) {
        if (s[j] == '6') continue;
        dp[i] += dp[j]; dp[i] %= mod;
        ans[i] += (1ll * 10 * ans[j] + 1ll * (s[i] - '0') * dp[j]) % mod; ans[i] %= mod;
        if (s[j] == s[i]) break;
    }
    res += ans[i]; res %= mod;
}
\end{lstlisting}

\end{minipage}
\end{frame}

% %%%%%%%%%%%%%%%%%%%%%%%%%%%%%%%%%%%%%%%%%%%%%%%%%%%%%%%%%%%%%%%%%%%%%%%%%%%%%
% 在这里填入题目
% %%%%%%%%%%%%%%%%%%%%%%%%%%%%%%%%%%%%%%%%%%%%%%%%%%%%%%%%%%%%%%%%%%%%%%%%%%%%%
\def\sectionName{最大异或值}



% 如果它是 beamer
\if 1\isBeamerMode\relax
    \section[\TOCName]{\sectionName}
\fi
% 如果它是 paper
\if 0\isBeamerMode\relax
    \section[\TOCName\ -\ \sectionName]{\sectionName}
\fi

\begin{frame}

% 如果它是 beamer
\if 1\isBeamerMode\relax
    {\Huge \sectionName}\par
\fi

% %%%%%%%%%%%%%%%%%%%%%%%%%%%%%%%%%%%%%%%%%%%%%%%%%%%%%%%%%%%%%%%%%%%%%%%%%%%%%
% 在这里填入你的名字
% %%%%%%%%%%%%%%%%%%%%%%%%%%%%%%%%%%%%%%%%%%%%%%%%%%%%%%%%%%%%%%%%%%%%%%%%%%%%%
\sectionAuthor{TTDragon}{dyyyyyyyy}



% %%%%%%%%%%%%%%%%%%%%%%%%%%%%%%%%%%%%%%%%%%%%%%%%%%%%%%%%%%%%%%%%%%%%%%%%%%%%%
% 这里可以写感想(嘲讽,bushi),也可以不写!!!
% %%%%%%%%%%%%%%%%%%%%%%%%%%%%%%%%%%%%%%%%%%%%%%%%%%%%%%%%%%%%%%%%%%%%%%%%%%%%%
思维题。



\end{frame}

% %%%%%%%%%%%%%%%%%%%%%%%%%%%%%%%%%%%%%%%%%%%%%%%%%%%%%%%%%%%%%%%%%%%%%%%%%%%%%
% 这里开始写简单的题目意思 ~
% %%%%%%%%%%%%%%%%%%%%%%%%%%%%%%%%%%%%%%%%%%%%%%%%%%%%%%%%%%%%%%%%%%%%%%%%%%%%%
\subsection{题目意思}
\begin{frame} % 如果一个 frame 写不下的话,多开几个就好了~
对于 $[1, n]$ 内整数 $i$ 和 $j$,求 $\max\{i \oplus j\}$。
\end{frame}



% %%%%%%%%%%%%%%%%%%%%%%%%%%%%%%%%%%%%%%%%%%%%%%%%%%%%%%%%%%%%%%%%%%%%%%%%%%%%%
% 这里开始写题解 ~
% %%%%%%%%%%%%%%%%%%%%%%%%%%%%%%%%%%%%%%%%%%%%%%%%%%%%%%%%%%%%%%%%%%%%%%%%%%%%%
\subsection{题解}
\begin{frame} % 如果一个 frame 写不下的话,多开几个就好了~
先说结论:
\begin{itemize}
    \item 记 $x$ 表示 $n$ 二进制的位数 (除去前导 $0$)。
    \item 答案是 $2^x - 1$,($n = 1$ 的情况需要特判)。
\end{itemize}
\end{frame}


\begin{frame}

首先证明 $2^x - 1$ 是可行的,因为 $n$ 是一个 $x$ 位二进制数。

对于 $n > 1$ 有:
\begin{itemize}
    \item $1\leq 2^{x - 1}\leq n$。
    \item $1\leq 2^{x - 1} - 1\leq n$。
    \item $2^{x - 1} \oplus (2^{x - 1} - 1) = 2^x - 1$。
\end{itemize}

然后我们证明不可能得出 $\geq 2^x$ 的答案。

对于一个 $\geq 2^x$ 的答案来说,其最高位的位数一定是 $> x$ 的,而对于所有在 $[1,
n]$ 以内的数,他们 $>x$ 二进制位上的值都为 $0$,显然无法得出 $\geq 2^x$ 的答案。

\end{frame}

% %%%%%%%%%%%%%%%%%%%%%%%%%%%%%%%%%%%%%%%%%%%%%%%%%%%%%%%%%%%%%%%%%%%%%%%%%%%%%
% 在这里填入题目
% %%%%%%%%%%%%%%%%%%%%%%%%%%%%%%%%%%%%%%%%%%%%%%%%%%%%%%%%%%%%%%%%%%%%%%%%%%%%%
\def\sectionName{happy 子序列}
\section{\sectionName}



\begin{frame}

% 如果它是 beamer
\if 1\isBeamerMode\relax
    {\Huge \sectionName}\par
\fi

% %%%%%%%%%%%%%%%%%%%%%%%%%%%%%%%%%%%%%%%%%%%%%%%%%%%%%%%%%%%%%%%%%%%%%%%%%%%%%
% 在这里填入你的名字
% %%%%%%%%%%%%%%%%%%%%%%%%%%%%%%%%%%%%%%%%%%%%%%%%%%%%%%%%%%%%%%%%%%%%%%%%%%%%%
\sectionAuthor{dyyyyyyyy}



% %%%%%%%%%%%%%%%%%%%%%%%%%%%%%%%%%%%%%%%%%%%%%%%%%%%%%%%%%%%%%%%%%%%%%%%%%%%%%
% 这里可以写感想(嘲讽,bushi),也可以不写!!!
% %%%%%%%%%%%%%%%%%%%%%%%%%%%%%%%%%%%%%%%%%%%%%%%%%%%%%%%%%%%%%%%%%%%%%%%%%%%%%
贪心 + 暴力。



\end{frame}

% %%%%%%%%%%%%%%%%%%%%%%%%%%%%%%%%%%%%%%%%%%%%%%%%%%%%%%%%%%%%%%%%%%%%%%%%%%%%%
% 这里开始写简单的题目意思 ~
% %%%%%%%%%%%%%%%%%%%%%%%%%%%%%%%%%%%%%%%%%%%%%%%%%%%%%%%%%%%%%%%%%%%%%%%%%%%%%
\subsection{题目意思}
\begin{frame} % 如果一个 frame 写不下的话,多开几个就好了~
求解字符串中是否存在一个子序列,等于给定的字符串。
\end{frame}



% %%%%%%%%%%%%%%%%%%%%%%%%%%%%%%%%%%%%%%%%%%%%%%%%%%%%%%%%%%%%%%%%%%%%%%%%%%%%%
% 这里开始写题解 ~
% %%%%%%%%%%%%%%%%%%%%%%%%%%%%%%%%%%%%%%%%%%%%%%%%%%%%%%%%%%%%%%%%%%%%%%%%%%%%%
\subsection{题解}
\begin{frame} % 如果一个 frame 写不下的话,多开几个就好了~
这里的匹配串 $T = \text{\cmd{"Carol"}}$,我们只需要判断 $T$ 是否在 $S$ 中出现就行。

那么一个很简单的贪心想法是,找到一个出现 $T_1$ 的位置 $p_1$,再向后找第一个出
现 $T_2$ 的位置 $p_2$……

就这样找下去,最后判断有没有找到就行。

时间复杂度:$O(|S||T|)$。

另外可以思考一下 $T$ 作为子序列的出现次数怎么求。
\end{frame}

% %%%%%%%%%%%%%%%%%%%%%%%%%%%%%%%%%%%%%%%%%%%%%%%%%%%%%%%%%%%%%%%%%%%%%%%%%%%%%
% 在这里填入题目
% %%%%%%%%%%%%%%%%%%%%%%%%%%%%%%%%%%%%%%%%%%%%%%%%%%%%%%%%%%%%%%%%%%%%%%%%%%%%%
\def\sectionName{Rabbit House的新菜单}



% 如果它是 beamer
\if 1\isBeamerMode\relax
    \section[\TOCName]{\sectionName}
\fi
% 如果它是 paper
\if 0\isBeamerMode\relax
    \section[\TOCName\ -\ \sectionName]{\sectionName}
\fi

\begin{frame}

% 如果它是 beamer
\if 1\isBeamerMode\relax
    {\Huge \sectionName}\par
\fi

% %%%%%%%%%%%%%%%%%%%%%%%%%%%%%%%%%%%%%%%%%%%%%%%%%%%%%%%%%%%%%%%%%%%%%%%%%%%%%
% 在这里填入你的名字
% %%%%%%%%%%%%%%%%%%%%%%%%%%%%%%%%%%%%%%%%%%%%%%%%%%%%%%%%%%%%%%%%%%%%%%%%%%%%%
\sectionAuthor{lzc2001, KryptonAu, ddlearn}{dyyyyyyyy}



% %%%%%%%%%%%%%%%%%%%%%%%%%%%%%%%%%%%%%%%%%%%%%%%%%%%%%%%%%%%%%%%%%%%%%%%%%%%%%
% 这里可以写感想(嘲讽,bushi),也可以不写!!!
% %%%%%%%%%%%%%%%%%%%%%%%%%%%%%%%%%%%%%%%%%%%%%%%%%%%%%%%%%%%%%%%%%%%%%%%%%%%%%
经典且比较裸的容斥原理题。



\end{frame}

% %%%%%%%%%%%%%%%%%%%%%%%%%%%%%%%%%%%%%%%%%%%%%%%%%%%%%%%%%%%%%%%%%%%%%%%%%%%%%
% 这里开始写简单的题目意思 ~
% %%%%%%%%%%%%%%%%%%%%%%%%%%%%%%%%%%%%%%%%%%%%%%%%%%%%%%%%%%%%%%%%%%%%%%%%%%%%%
\subsection{题目意思}
\begin{frame} % 如果一个 frame 写不下的话,多开几个就好了~
给定 $n$ 个菜品,其中第 $i$ 个有两个属性值 $a_i$ 和 $b_i$。

我们需要计算满足下列条件的组合数:
\begin{enumerate}
    \item 选择 $3$ 个不同的菜品。
    \item 菜品的 $a_i$ 两两不相同或者 $b_i$ 两两不相同。
\end{enumerate}
\end{frame}



% %%%%%%%%%%%%%%%%%%%%%%%%%%%%%%%%%%%%%%%%%%%%%%%%%%%%%%%%%%%%%%%%%%%%%%%%%%%%%
% 这里开始写题解 ~
% %%%%%%%%%%%%%%%%%%%%%%%%%%%%%%%%%%%%%%%%%%%%%%%%%%%%%%%%%%%%%%%%%%%%%%%%%%%%%
\subsection{题解}


\begin{frame} % 如果一个 frame 写不下的话,多开几个就好了~
我们需要记录三个东西:

\begin{enumerate}
    \item \cmd{map<pair<int, int>, int> cnt}:\cmd{cnt[\{x, y\}]} 表示点对 $(x,
        y)$ 的菜品有多少个
    \item \cmd{map<int, int> cnta}:\cmd{cnta[x]} 表示 $a_i = x$ 的 菜品有多少个
    \item \cmd{map<int, int> cntb}:\cmd{cntb[x]} 表示 $b_i = x$ 的 菜品有多少个
\end{enumerate}
\end{frame}


\begin{frame} % 如果一个 frame 写不下的话,多开几个就好了~

如果不考虑约束条件,所有的选择方法为 ${n \choose 3}$。

先在我们考虑计算不满足约束条件的情况 $ans_v$。

有哪些情况不满足呢?

对于每一种 $(a_i, b_i)$ 来说,
\begin{itemize}
    \item 为方便表示,令:$v =\ $\cmd{cnt[\{$a_i$, $b_i$\}]},$x =\ $
        \cmd{cnta[$a_i$]},$y =\ $\cmd{cntb[$b_i$]}。
	\item 三个菜品的美味度和观赏度都一样,有 ${v \choose 3}$ 种。
	\item 两个菜品的美味度和观赏度都一样,有 ${v \choose 2} \times (n - v)$ 种。
	\item 两个菜品的美味度一样,两个菜品的观赏度一样,有 ${v\times (x - v)\times
        (y - v)}$ 种。
\end{itemize}

用总情况减去不满足约束的情况就是满足约束的情况。

时间复杂度:$O(N\log N)$。
\end{frame}



% %%%%%%%%%%%%%%%%%%%%%%%%%%%%%%%%%%%%%%%%%%%%%%%%%%%%%%%%%%%%%%%%%%%%%%%%%%%%%
% 在这里填入题目
% %%%%%%%%%%%%%%%%%%%%%%%%%%%%%%%%%%%%%%%%%%%%%%%%%%%%%%%%%%%%%%%%%%%%%%%%%%%%%
\def\sectionName{Peter 和他的王国 1}
\section[\TOCName]{\sectionName}



\begin{frame}

% 如果它是 beamer
\if 1\isBeamerMode\relax
    {\Huge \sectionName}\par
\fi

% %%%%%%%%%%%%%%%%%%%%%%%%%%%%%%%%%%%%%%%%%%%%%%%%%%%%%%%%%%%%%%%%%%%%%%%%%%%%%
% 在这里填入你的名字
% %%%%%%%%%%%%%%%%%%%%%%%%%%%%%%%%%%%%%%%%%%%%%%%%%%%%%%%%%%%%%%%%%%%%%%%%%%%%%
\sectionAuthor{Peterlits Zo}{Peterlits Zo}



% %%%%%%%%%%%%%%%%%%%%%%%%%%%%%%%%%%%%%%%%%%%%%%%%%%%%%%%%%%%%%%%%%%%%%%%%%%%%%
% 这里可以写感想(嘲讽,bushi),也可以不写!!!
% %%%%%%%%%%%%%%%%%%%%%%%%%%%%%%%%%%%%%%%%%%%%%%%%%%%%%%%%%%%%%%%%%%%%%%%%%%%%%
道理我都懂,不过为啥这个题目后面有一个数字 1 呢?



\end{frame}

% %%%%%%%%%%%%%%%%%%%%%%%%%%%%%%%%%%%%%%%%%%%%%%%%%%%%%%%%%%%%%%%%%%%%%%%%%%%%%
% 这里开始写简单的题目意思 ~
% %%%%%%%%%%%%%%%%%%%%%%%%%%%%%%%%%%%%%%%%%%%%%%%%%%%%%%%%%%%%%%%%%%%%%%%%%%%%%
\subsection{题目意思}
\begin{frame} % 如果一个 frame 写不下的话,多开几个就好了~
给定一个数组 $A$,第 $i$ 个元素为 $A_i$。

构造一个集合,即:\[\{x \mid A_i \in A, A_j \in A, x \mid A_i, x \mid A_j, x \le
l\}\]求该集合中的最大值。
\end{frame}



% %%%%%%%%%%%%%%%%%%%%%%%%%%%%%%%%%%%%%%%%%%%%%%%%%%%%%%%%%%%%%%%%%%%%%%%%%%%%%
% 这里开始写题解 ~
% %%%%%%%%%%%%%%%%%%%%%%%%%%%%%%%%%%%%%%%%%%%%%%%%%%%%%%%%%%%%%%%%%%%%%%%%%%%%%
\subsection{题解}
\begin{frame} % 如果一个 frame 写不下的话,多开几个就好了~
我们知道,最大公约数能够被其他所有的公约数整除,那么如果我们求出最大公因数的所有
因数,然后二分即可得到在 $[1, l]$ 区间内的公约数。时间复杂度为 $O(\sqrt P)$。其
中分解质因数的时间复杂度是 $O(\sqrt P)$,而二分查找是 $O(\log \sqrt P)$,可以忽
略不计。

总复杂度为两两组合乘以上面的复杂度,故为 $O(n^2 \sqrt P)$。
\end{frame}



% %%%%%%%%%%%%%%%%%%%%%%%%%%%%%%%%%%%%%%%%%%%%%%%%%%%%%%%%%%%%%%%%%%%%%%%%%%%%%
% 在这里 include 所有参考代码
% %%%%%%%%%%%%%%%%%%%%%%%%%%%%%%%%%%%%%%%%%%%%%%%%%%%%%%%%%%%%%%%%%%%%%%%%%%%%%
% 如果它是 paper
\if 0\isBeamerMode\relax
    \section{参考代码}
    \addCode{title}
\fi



\end{document}
