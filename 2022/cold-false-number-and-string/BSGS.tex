
% %%%%%%%%%%%%%%%%%%%%%%%%%%%%%%%%%%%%%%%%%%%%%%%%%%%%%%%%%%%%%%%%%%%%%%%%%%%%%
% 在这里填入题目
% %%%%%%%%%%%%%%%%%%%%%%%%%%%%%%%%%%%%%%%%%%%%%%%%%%%%%%%%%%%%%%%%%%%%%%%%%%%%%
\def\sectionName{数论 - BSGS}



% 如果它是 beamer
\if 1\isBeamerMode\relax
    \section[\TOCName]{\sectionName}
\fi
% 如果它是 paper
\if 0\isBeamerMode\relax
    \section[\TOCName\ -\ \sectionName]{\sectionName}
\fi

\begin{frame}

% 如果它是 beamer
\if 1\isBeamerMode\relax
    \noindent{\Huge \sectionName}\par
\fi

% %%%%%%%%%%%%%%%%%%%%%%%%%%%%%%%%%%%%%%%%%%%%%%%%%%%%%%%%%%%%%%%%%%%%%%%%%%%%%
% 在这里填入你的名字
% %%%%%%%%%%%%%%%%%%%%%%%%%%%%%%%%%%%%%%%%%%%%%%%%%%%%%%%%%%%%%%%%%%%%%%%%%%%%%
\sectionAuthor{Peterlits Zo}



% %%%%%%%%%%%%%%%%%%%%%%%%%%%%%%%%%%%%%%%%%%%%%%%%%%%%%%%%%%%%%%%%%%%%%%%%%%%%%
% 这里可以写感想(嘲讽,bushi),也可以不写!!!
% %%%%%%%%%%%%%%%%%%%%%%%%%%%%%%%%%%%%%%%%%%%%%%%%%%%%%%%%%%%%%%%%%%%%%%%%%%%%%
\noindent 用在区间里会很快哦。



\end{frame}

% %%%%%%%%%%%%%%%%%%%%%%%%%%%%%%%%%%%%%%%%%%%%%%%%%%%%%%%%%%%%%%%%%%%%%%%%%%%%%
% 取模
% %%%%%%%%%%%%%%%%%%%%%%%%%%%%%%%%%%%%%%%%%%%%%%%%%%%%%%%%%%%%%%%%%%%%%%%%%%%%%
\subsection{取模}

\begin{frame}
我们在快速幂中讲到了,对于 $a^b$ 这种幂运算而言,我们可以在 $O(\lg b)$ 的时间复
杂度内解得结果。

但是有一些题目要求我们能在 $O(1)$ 的时间复杂度来得到。这可不简单哇。
\end{frame}

\begin{frame}
朴素的思想是打表。假设我们的 $b$ 的上限是 $M$,那么我们只要在 $O(M)$ 的时间复杂
度内求得 $i \in [1, M]$ 的 $a^i$ 对应值。\pause

那么我们要求 $a^b$ 直接在表中拿第 $i$ 个就行!!
\end{frame}

\begin{frame}
这种方法的确不错。但是如果 $M$ 太大了就没啥用了。

我们不妨这么想,对于任意一个数 $a^b$ 而言,我们都有,存在 $c$ 和 $d$,使得:$a^b
= a^{c \sqrt M + d}$,其中 $0 \le c < \sqrt M$,而 $0 \le d < \sqrt M$。\pause

那么我们只需要打表 $a^{c \sqrt M}, 0 \le c < \sqrt M$,和 $a^{d}, 0 \le d <
\sqrt M$,我们先算出来 $c$ 和 $d$,再查两次表即可。\pause

可以看到,搞表只要 $O(\sqrt M)$ 的时间/空间复杂度。而我们得到它也是 $O(1)$ 的时
间。
\end{frame}

\begin{frame}
BSGS 的意思是 baby step giant step。这也就是说,我们先 $\sqrt M$ 地移动,移动完
了之后再 $1$ 地移动,即可。
\end{frame}

\begin{frame}
举例子,我们要求 $a^5$ 和 $a^{18}$ 和 $a^{24}$,可以看到我们可以取 $M = 25$,那
么有 $\sqrt M = 5$,令 $a = 2$。\pause

那么有:
\begin{center}
\begin{tabular}{lrrrrr}
    \toprule
    $i$               & $0$ & $1$  & $2$    & $3$     & $4$       \\
    \midrule
    $a^{i \sqrt M}$   & $1$ & $32$ & $1024$ & $32768$ & $1048576$ \\
    $a^{i}$           & $1$ & $2$  & $4$    & $8$     & $16$      \\
    \bottomrule
\end{tabular}
\end{center} \pause

同时,我们有:\pause
\begin{itemize}
    \item $a^5 = a^{1 \times 5 + 0} = a^{1 \sqrt M} \times a^{0} = 32$。\pause
    \item $a^{18} = a^{3 \times 5 + 3} = a^{3 \sqrt M} \times a^{3}$。\pause
    \item $a^{24} = a^{4 \times 5 + 4} = a^{4 \sqrt M} \times a^{4}$。
\end{itemize}
\end{frame}

\begin{frame}
有几个题目:
\begin{itemize}
    \item LiberOJ - p162。
    \item Luogu - P1226。
\end{itemize}
\end{frame}
