
% %%%%%%%%%%%%%%%%%%%%%%%%%%%%%%%%%%%%%%%%%%%%%%%%%%%%%%%%%%%%%%%%%%%%%%%%%%%%%
% 在这里填入题目
% %%%%%%%%%%%%%%%%%%%%%%%%%%%%%%%%%%%%%%%%%%%%%%%%%%%%%%%%%%%%%%%%%%%%%%%%%%%%%
\def\sectionName{数论 - 快速幂}



% 如果它是 beamer
\if 1\isBeamerMode\relax
    \section[\TOCName]{\sectionName}
\fi
% 如果它是 paper
\if 0\isBeamerMode\relax
    \section[\TOCName\ -\ \sectionName]{\sectionName}
\fi

\begin{frame}

% 如果它是 beamer
\if 1\isBeamerMode\relax
    \noindent{\Huge \sectionName}\par
\fi

% %%%%%%%%%%%%%%%%%%%%%%%%%%%%%%%%%%%%%%%%%%%%%%%%%%%%%%%%%%%%%%%%%%%%%%%%%%%%%
% 在这里填入你的名字
% %%%%%%%%%%%%%%%%%%%%%%%%%%%%%%%%%%%%%%%%%%%%%%%%%%%%%%%%%%%%%%%%%%%%%%%%%%%%%
\sectionAuthor{Peterlits Zo}



% %%%%%%%%%%%%%%%%%%%%%%%%%%%%%%%%%%%%%%%%%%%%%%%%%%%%%%%%%%%%%%%%%%%%%%%%%%%%%
% 这里可以写感想(嘲讽,bushi),也可以不写!!!
% %%%%%%%%%%%%%%%%%%%%%%%%%%%%%%%%%%%%%%%%%%%%%%%%%%%%%%%%%%%%%%%%%%%%%%%%%%%%%
\noindent 快速幂,xiuxiuxiu~



\end{frame}

% %%%%%%%%%%%%%%%%%%%%%%%%%%%%%%%%%%%%%%%%%%%%%%%%%%%%%%%%%%%%%%%%%%%%%%%%%%%%%
% 逻辑代数
% %%%%%%%%%%%%%%%%%%%%%%%%%%%%%%%%%%%%%%%%%%%%%%%%%%%%%%%%%%%%%%%%%%%%%%%%%%%%%
\subsection{引入}

\begin{frame}
我们在做数论题目的时候,经常需要计算 $a^b \bmod m$ 的值。\pause

在朴素算法下,我们很自然就会想到 $a \times a \times \cdots \times a \bmod m$。这
个复杂度是 $O(b)$ 的。\pause

而且不仅如此,它还会造成在取模之前就溢出了,从而导致答案不正确。根据模运算的性质,
我们可以得到 $(\cdots ((a \times a \bmod m) \times a \bmod m) \cdots)$,这在一定
程度上可以避免答案不正确,但是还是带来了超级大的复杂度。
\end{frame}

\begin{frame}
就像基础思想倍增一样,我们注意到:
\begin{itemize}
    \item 一旦我们得到了 $a$,我们就能立即得到 $a^2$。\pause
    \item 一旦我们得到了 $a^2$,我们就能立即得到 $a^4$。\pause
    \item 一旦我们得到了 $a^4$,我们就能立即得到 $a^8$。\pause
    \item 一旦我们得到了 $a^{i}$,我们就能立即得到 $a^{2i}$。\pause
\end{itemize}

而好巧不巧,我们能把一个 $a^b$ 分解为(其中 $f(0, x) = 1$,而 $f(1, x) = x$): \[
    f(b_1, a^1) \times f(b_2, a^2) \times f(b_4, a^4) \times f(b_8, a^8) \times \cdots
\]而其中 $b_{2^i} \in \{0, 1\}$。把 $b_{2^i}$ 按照 $i$ 放在一起,真好就是 $b$ 的
二进制形式。
\end{frame}

\subsection{递归版本}
\begin{frame}
比如说,我们想要计算 $a^{77}$,其中 $77 = (1001101)_2$,同时我们有:\[
    a^{77} = f(1, a^{64}) \times f(0, a^{32}) \times f(0, a^{16}) \times f(1,
        a^{8}) \times f(1, a^{4}) \times f(0, a^{2}) \times f(1, a)
\]\pause

那很明显,如果我们定义一个函数 $p(a, b) = a^b$,那么有:\[
    p(a, b) = \begin{cases}
        1,                                & b = 0, \\
        p(a, {b \over 2})^2,              & b \not= 0 \land b \bmod 2 = 0,\\
        p(a, {b - 1 \over 2})^2 \times a, & b \not= 0 \land b \bmod 2 = 1.\\
    \end{cases}
\]
\end{frame}

\begin{frame}[fragile]
因此我们可以写一个这个代码:
\begin{lstlisting}[language=C++]
int p(int a, int b) {
    if (b == 0) {
        return 1;
    } else if (b % 2 == 0) { // !(b & 1)
        int result = p(a, b / 2);
        return result * result;
    } else {
        int result = p(a, b / 2);
        return result * result * a;
    }
}
\end{lstlisting}
\end{frame}

\begin{frame}
很明显,无论 $b$ 是偶数还是奇数,我们需要解的子问题只有一个,并且子问题的规模是
原来的一半。\pause

我们不妨假设我们的问题规模是 $n = 77$,那么我们的子问题对应的问题规模所对应的链
为:\[
    77 \to 38 \to 19 \to 9 \to 4 \to 2 \to 1
\]也就是说,我们基本上要递归 $\log_2 77 \simeq 7$ 层,每一层的时间复杂度均为
$O(1)$,总时间复杂度为 $O(log_2 n)$。
\end{frame}

\subsection{循环版本}
\begin{frame}[fragile]
但是一般而言,我们都更喜欢循环而不是递归的解法(除非递归的解法实在是太简单了)。
那么对于循环的解法呢?

我们或许需要维护 $a^{2^i}$,并且根据 $b$ 的第 $i$ 位究竟是 $0$ 还是 $1$ 来乘入到
最后的结果中。

\begin{lstlisting}[language=C++]
int fast_pow(int a, int b) {
    int result = 1; // 一开始为单位元 1。
    while (b) { // 当 b 还可以压榨的话,就压榨一哈。
        // 实际上,a 代表的是 a^{2^i}。一开始是 a。
        if (b & 1) result *= a;
        a *= a; // a^{2^i} -> a^{2^{i + 1}}
        b >>= 1; // 最后一位压榨完了,已经没用了。
    }
    return result;
}
\end{lstlisting}
\end{frame}

\begin{frame}[fragile]
还记得之前说过的,我们需要利用模运算的性质,把外面的模运算搞到里面去,避免溢出吗?
这里的话,避免溢出,一是仔细思考到底需要啥类型(对,你可能需要的不是 \cmd{int}),二是:
\begin{lstlisting}[language=C++]
int fast_pow(int a, int b, int MOD) {
    int result = 1;
    while (b) {
        if (b & 1)
            result = result * 1ll * a % MOD;
        a = a * 1ll * a % MOD;
        b >>= 1;
    }
    return result;
}
\end{lstlisting}
\end{frame}

\begin{frame}
接下来是和快速幂关系很大的 BSGS。和快速幂的 $O(\lg n)$ 不同,BSGS 的复杂度是
$O((1)$,当然对应着,它也会付出相应的代价,嘻嘻嘻~
\end{frame}
