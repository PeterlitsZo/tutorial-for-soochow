% %%%%%%%%%%%%%%%%%%%%%%%%%%%%%%%%%%%%%%%%%%%%%%%%%%%%%%%%%%%%%%%%%%%%%%%%%%%%%
% 选择 isBeamerMode -> 0: 文件,1:幻灯片
% %%%%%%%%%%%%%%%%%%%%%%%%%%%%%%%%%%%%%%%%%%%%%%%%%%%%%%%%%%%%%%%%%%%%%%%%%%%%%
\def\isBeamerMode{1}
\if 0\isBeamerMode
    \makeatletter
    \newif\ifntx@origotf
    \makeatother
    \PassOptionsToPackage{nofontspec}{newtxtext}
    \documentclass[lang=cn]{elegantpaper}
    \usepackage{xcolor}
    \renewenvironment{frame}[1][]{}{}
    \setcounter{tocdepth}{1}
\else
    \documentclass{ctexbeamer}
    \usepackage{listings}
    \usepackage{xcolor}
    \usepackage{multicol}
    % 幻灯片的一些配置
    \usetheme{Berlin}
    \AtBeginSection[] {
        \begin{frame}
            \begin{multicols}{2}
            \tableofcontents[currentsection, hideallsubsections]
            \end{multicols}
        \end{frame}
    }
\fi



% %%%%%%%%%%%%%%%%%%%%%%%%%%%%%%%%%%%%%%%%%%%%%%%%%%%%%%%%%%%%%%%%%%%%%%%%%%%%%
% 一些宏定义和相关配置
% %%%%%%%%%%%%%%%%%%%%%%%%%%%%%%%%%%%%%%%%%%%%%%%%%%%%%%%%%%%%%%%%%%%%%%%%%%%%%
% 让 \normalsizeHeight 为汉字的字宽
\makeatletter
\normalsize
\newlength{\normalsizeHeight}
\setlength{\normalsizeHeight}{\f@size pt}
\makeatother


% \sectionAuthor{<出题人姓名>}
% 用于添加章节作者
\newcommand\sectionAuthor[1]{%
    % style at begin
    \if 0\isBeamerMode\relax%
        \small\color{gray}%
        \vspace*{-1cm}\hfill%
    \else%
        \vspace*{0.3cm}%
    \fi%
    % text...
    Tutorial by \textbf{#1}.%
    \if 1\isBeamerMode\relax%
        \par%
    \fi%
    % style at end
    \if 0\isBeamerMode\relax%
        \normalsize\color{black}%
        \vspace*{0.3cm}%
    \else%
        \vspace*{0.5cm}%
    \fi}


% \addCode{<章节名>}{<代码文件名>}{<高亮名>}
% e.g. \addCode{小明和他的朋友}{xiaomi_and_his_friend.cpp}{cpp}
% 用于陈列代码
\newcommand\addCode[3]{%
    \subsection{#1}%
    \lstinputlisting[language=#3]{#2}}


% \cmd{<代码>}
% 用于陈列行内代码
\newcommand\cmd[1]{%
    \texttt{#1}}


% 配置陈列代码的风格
\lstset{
    xleftmargin         =   2\normalsizeHeight, % 缩进
    basicstyle          =   \ttfamily,          % 基本代码风格
    keywordstyle        =   \bfseries,          % 关键字风格
    commentstyle        =   \rmfamily\itshape,  % 注释的风格,斜体
    stringstyle         =   \ttfamily,          % 字符串风格
    flexiblecolumns,                            % 别问为什么,加上这个
    numbers             =   left,               % 行号的位置在左边
    showspaces          =   false,              % 是否显示空格,显示了有点乱,所以不现实了
    numberstyle         =   \ttfamily,          % 行号的样式,小五号,tt等宽字体
    showstringspaces    =   false,
    captionpos          =   t,                  % 这段代码的名字所呈现的位置,t指的是top上面
    frame               =   lrtb,               % 显示边框
    columns             =   fixed,              % 字的宽度固定
    basewidth           =   0.5em,              % 固定为 0.5em
    tabsize             =   4,                  % tab 展示为 4 个空格宽度
    keywordstyle        =   \color{blue},
    keywordstyle        =   [2] \color{teal},
    stringstyle         =   \color{magenta},
    commentstyle        =   \color{red}\ttfamily,
}



% %%%%%%%%%%%%%%%%%%%%%%%%%%%%%%%%%%%%%%%%%%%%%%%%%%%%%%%%%%%%%%%%%%%%%%%%%%%%%
% 在这里编辑基础信息
% %%%%%%%%%%%%%%%%%%%%%%%%%%%%%%%%%%%%%%%%%%%%%%%%%%%%%%%%%%%%%%%%%%%%%%%%%%%%%
\title{寒假集训 - 数论和字符串}
\author{苏州大学 ACM 集训队}



\begin{document}

% %%%%%%%%%%%%%%%%%%%%%%%%%%%%%%%%%%%%%%%%%%%%%%%%%%%%%%%%%%%%%%%%%%%%%%%%%%%%%
% 生成标题
% %%%%%%%%%%%%%%%%%%%%%%%%%%%%%%%%%%%%%%%%%%%%%%%%%%%%%%%%%%%%%%%%%%%%%%%%%%%%%
% 如果它是 paper
\if 0\isBeamerMode\relax
    \maketitle
    \tableofcontents
    \newpage
\fi

% 如果它是 beamer
\if 1\isBeamerMode\relax
    \frame{\titlepage}
\fi



% %%%%%%%%%%%%%%%%%%%%%%%%%%%%%%%%%%%%%%%%%%%%%%%%%%%%%%%%%%%%%%%%%%%%%%%%%%%%%
% 在这里 include 所有题解
% %%%%%%%%%%%%%%%%%%%%%%%%%%%%%%%%%%%%%%%%%%%%%%%%%%%%%%%%%%%%%%%%%%%%%%%%%%%%%
\def\TOCName{位运算} 
% %%%%%%%%%%%%%%%%%%%%%%%%%%%%%%%%%%%%%%%%%%%%%%%%%%%%%%%%%%%%%%%%%%%%%%%%%%%%%
% 在这里填入题目
% %%%%%%%%%%%%%%%%%%%%%%%%%%%%%%%%%%%%%%%%%%%%%%%%%%%%%%%%%%%%%%%%%%%%%%%%%%%%%
\def\sectionName{数论 - 位运算}



% 如果它是 beamer
\if 1\isBeamerMode\relax
    \section[\TOCName]{\sectionName}
\fi
% 如果它是 paper
\if 0\isBeamerMode\relax
    \section[\TOCName\ -\ \sectionName]{\sectionName}
\fi

\begin{frame}

% 如果它是 beamer
\if 1\isBeamerMode\relax
    {\Huge \sectionName}\par
\fi

% %%%%%%%%%%%%%%%%%%%%%%%%%%%%%%%%%%%%%%%%%%%%%%%%%%%%%%%%%%%%%%%%%%%%%%%%%%%%%
% 在这里填入你的名字
% %%%%%%%%%%%%%%%%%%%%%%%%%%%%%%%%%%%%%%%%%%%%%%%%%%%%%%%%%%%%%%%%%%%%%%%%%%%%%
\sectionAuthor{Peterlits Zo}



% %%%%%%%%%%%%%%%%%%%%%%%%%%%%%%%%%%%%%%%%%%%%%%%%%%%%%%%%%%%%%%%%%%%%%%%%%%%%%
% 这里可以写感想(嘲讽,bushi),也可以不写!!!
% %%%%%%%%%%%%%%%%%%%%%%%%%%%%%%%%%%%%%%%%%%%%%%%%%%%%%%%%%%%%%%%%%%%%%%%%%%%%%
位运算很巧妙的~



\end{frame}

% %%%%%%%%%%%%%%%%%%%%%%%%%%%%%%%%%%%%%%%%%%%%%%%%%%%%%%%%%%%%%%%%%%%%%%%%%%%%%
% 逻辑代数
% %%%%%%%%%%%%%%%%%%%%%%%%%%%%%%%%%%%%%%%%%%%%%%%%%%%%%%%%%%%%%%%%%%%%%%%%%%%%%
\subsection{逻辑代数}
\begin{frame} % 如果一个 frame 写不下的话,多开几个就好了~
逻辑代数中只有两个元素,既真和假。举例来说的话,我们有:
\begin{itemize}
    \item 太阳从东边升起。\pause $\to T$。
    \item $1 + 1 = 3$。\pause $\to F$。
\end{itemize}

可以看到命题可以得到它的真值。接下来我们要定义它们的运算。
\end{frame}

\begin{frame}
一般而言我们有四个运算:
\begin{itemize}
    \item 或。两个中有至少一个为真它就是真。我们定义符号有 $\lor$。\pause
    \item 与。两个中同时为真它才是真。我们定义符号有 $\land$。\pause
    \item 非。它能把真的变成假的,把假的变成真的。我们定义符号有 $\lnot$。\pause
    \item 亦或。两个中\textbf{有且仅有}一个为真它才是真。我们定义符号有 $\oplus$。
\end{itemize}
\end{frame}

\begin{frame}
\begin{block}{符号表}
或:$\lor$\hfill 与:$\land$\hfill 非:$\lnot$\hfill 亦或:$\oplus$ \hfill
\end{block}

用抽象后的逻辑代数可以帮助我们更好的理解和运算。

比如:如果太阳从东边升起,或者,太阳从西边升起并且 $1 + 1 = 3$ 的时候,Peterlits
就会摆烂。我们希望得知 Peterlits 到底会不会摆烂。那么它对应的表达式就是:\pause

\[T \lor (F \land F)\]

运算可得,其结果是 $T$。
\end{frame}

\begin{frame}
\begin{block}{符号表}
或:$\lor$\hfill 与:$\land$\hfill 非:$\lnot$\hfill 亦或:$\oplus$ \hfill
\end{block}

因为在 ASCII 下不太好输入,C++ 用以下的符号来代表逻辑代数的运算:
\begin{itemize}
    \item 或。\cmd{||}。\pause
    \item 与。\cmd{\&\&}。\pause
    \item 非。\cmd{!}。\pause
    \item 亦或。嘻嘻,C++ 没有这个。
\end{itemize}
\end{frame}

\begin{frame}[fragile]
\begin{block}{符号表}
或:$\lor$ / \cmd{||}\hfill
与:$\land$ / \cmd{\&\&}\hfill
非:$\lnot$ / \cmd{!}\hfill
亦或:$\oplus$ \hfill
\end{block}

\begin{lstlisting}[language=C++]
int foo = 10;
int bar = 30;

if (foo < bar || false) {
    printf("ooooops!!"); // 这个一定会输出的!
}
\end{lstlisting}
\end{frame}

\begin{frame}[fragile]
在 C++ 中,我们认为,一切非零的都是真的($T$),而零是假的($F$)。\pause

也就是说下面的我们会得到输出 \cmd{1}:
\begin{lstlisting}[language=C++]
// 不过用数字来做逻辑运算太少见,所以应该会 WARNING。
printf("%d", (100 || 0));
\end{lstlisting}
\end{frame}

\begin{frame}
那么位运算就是我们上面讲的吗?不是~不过有一些关系的。
\end{frame}


% %%%%%%%%%%%%%%%%%%%%%%%%%%%%%%%%%%%%%%%%%%%%%%%%%%%%%%%%%%%%%%%%%%%%%%%%%%%%%
% 位运算
% %%%%%%%%%%%%%%%%%%%%%%%%%%%%%%%%%%%%%%%%%%%%%%%%%%%%%%%%%%%%%%%%%%%%%%%%%%%%%
\subsection{位运算}
\begin{frame}
\begin{block}{符号表}
或:$\lor$ / \cmd{||}\hfill
与:$\land$ / \cmd{\&\&}\hfill
非:$\lnot$ / \cmd{!}\hfill
亦或:$\oplus$ \hfill
\end{block}
我们知道,计算机内的一些东西都是二进制的,无论是数字、字符还是字符串,都是二进制
的。如果能把二进制映射到 $T/F$ 上就好了:\pause
\begin{itemize}
    \item $1 \to T$。\pause
    \item $0 \to F$。\pause
\end{itemize}

根据这个,我们自然而然的就得到了位运算了:数值都是由 $0 / 1$ 组成的,那么我们让
对应的值进行逻辑运算,一一收集起来就可以得到结果了。
\end{frame}

\begin{frame}
\begin{block}{符号表}
或:$\lor$ / \cmd{||}\hfill
与:$\land$ / \cmd{\&\&}\hfill
非:$\lnot$ / \cmd{!}\hfill
亦或:$\oplus$ \hfill
\end{block}

我们定义位运算如下:
\begin{itemize}
    \item 按位与。\cmd{\&}。
    \item 按位或。\cmd{|}。
    \item 按位取反。\cmd{\ \~}。
    \item 按位亦或。\cmd{\ \^}。
\end{itemize}

有的时候,我们说某些操作的时候,可能会省略掉\textbf{按位}两个字,这种时候就需要
自己判断到底说的是逻辑运算还是位运算了。
\end{frame}

\begin{frame}[fragile]
\begin{block}{符号表}
或:$\lor$ / \cmd{||} / \cmd{|}(按位)\hfill
与:$\land$ / \cmd{\&\&} / \cmd{\&}(按位)\hfill

非:$\lnot$ / \cmd{!} / \cmd{\ \~}(按位)\hfill
亦或:$\oplus$ / \cmd{\ \^}(按位) \hfill
\end{block}

比如说我们可以来模拟一哈按位亦或。我们知道,亦或的结果为真,当且仅当一个为真。

也就是说,我们有:
\begin{lstlisting}[language=C++]
int a = 0 ^ 0; \\ a = 0,因为只有 0 个 1。
int b = 0 ^ 1; \\ b = 1,因为正好有 1 个 1。
int c = 1 ^ 0; \\ c = 1,因为正好有 1 个 1。
int d = 1 ^ 1; \\ c = 2,因为有 2 个 1。
\end{lstlisting}
\end{frame}

\begin{frame}
现在我们希望得到(下列数字均在表示在二进制下)下述表达式的运算结果:\[
    1000100101110 \text{\cmd{\ \^}} 1010100011010
\] \pause

那么有:
\begin{center}
\begin{tabular}{lrrrrrrrrrrrrr}
    \toprule
    $a$  & $1$ & $0$ & $0$ & $0$ & $1$ & $0$ & $0$ & $1$ & $0$ & $1$ & $1$ & $1$ & $0$ \\
    $b$  & $1$ & $0$ & $1$ & $0$ & $1$ & $0$ & $0$ & $0$ & $1$ & $1$ & $0$ & $1$ & $0$ \\
    \midrule
    结果 & $0$ & $0$ & $1$ & $0$ & $0$ & $0$ & $0$ & $1$ & $1$ & $0$ & $1$ & $0$ & $0$ \\
    \bottomrule
\end{tabular}
\end{center}
\end{frame}

\begin{frame}
\begin{block}{符号表}
或:$\lor$ / \cmd{||} / \cmd{|}(按位)\hfill
与:$\land$ / \cmd{\&\&} / \cmd{\&}(按位)\hfill

非:$\lnot$ / \cmd{!} / \cmd{\ \~}(按位)\hfill
亦或:$\oplus$ / \cmd{\ \^}(按位) \hfill
\end{block}

除了符号表里的东西,我们还有其他的运算符:\cmd{>>} 和 \cmd{<<}。

它可以让一个数在二进制下左移或者右移。
\end{frame}

\begin{frame}
我们注意到,一个数一旦左移一位的话,随之带来的就是这个数会变大为原来的两倍。那么
我们就可以用 \cmd{1 << n} 来表示 $2^n$\footnote{这里和下面的东西,均假设我们的被
操作数是一个正数}。

同理,我们可以用右移来表示除以 $2$ 并向下取整。
\end{frame}

\begin{frame}
我们发现了一个数是偶数,当且仅当二进制下最后一个位为 $0$,那么我们要是能够把这玩
意取出来就好了。\pause

我们可以使用 \cmd{a \& 1} 这个位运算。
\end{frame}

\begin{frame}
当然位运算的一些小技巧还有很多都没见。不过讲快速幂肯定是绰绰有余了。
\end{frame}

% \def\TOCName{数论 - 快速幂} 
% %%%%%%%%%%%%%%%%%%%%%%%%%%%%%%%%%%%%%%%%%%%%%%%%%%%%%%%%%%%%%%%%%%%%%%%%%%%%%
% 在这里填入题目
% %%%%%%%%%%%%%%%%%%%%%%%%%%%%%%%%%%%%%%%%%%%%%%%%%%%%%%%%%%%%%%%%%%%%%%%%%%%%%
\def\sectionName{数论 - 快速幂}



% 如果它是 beamer
\if 1\isBeamerMode\relax
    \section[\TOCName]{\sectionName}
\fi
% 如果它是 paper
\if 0\isBeamerMode\relax
    \section[\TOCName\ -\ \sectionName]{\sectionName}
\fi

\begin{frame}

% 如果它是 beamer
\if 1\isBeamerMode\relax
    \noindent{\Huge \sectionName}\par
\fi

% %%%%%%%%%%%%%%%%%%%%%%%%%%%%%%%%%%%%%%%%%%%%%%%%%%%%%%%%%%%%%%%%%%%%%%%%%%%%%
% 在这里填入你的名字
% %%%%%%%%%%%%%%%%%%%%%%%%%%%%%%%%%%%%%%%%%%%%%%%%%%%%%%%%%%%%%%%%%%%%%%%%%%%%%
\sectionAuthor{Peterlits Zo}



% %%%%%%%%%%%%%%%%%%%%%%%%%%%%%%%%%%%%%%%%%%%%%%%%%%%%%%%%%%%%%%%%%%%%%%%%%%%%%
% 这里可以写感想(嘲讽,bushi),也可以不写!!!
% %%%%%%%%%%%%%%%%%%%%%%%%%%%%%%%%%%%%%%%%%%%%%%%%%%%%%%%%%%%%%%%%%%%%%%%%%%%%%
\noindent 快速幂,xiuxiuxiu~



\end{frame}

% %%%%%%%%%%%%%%%%%%%%%%%%%%%%%%%%%%%%%%%%%%%%%%%%%%%%%%%%%%%%%%%%%%%%%%%%%%%%%
% 逻辑代数
% %%%%%%%%%%%%%%%%%%%%%%%%%%%%%%%%%%%%%%%%%%%%%%%%%%%%%%%%%%%%%%%%%%%%%%%%%%%%%
\begin{frame}
\end{frame}




% %%%%%%%%%%%%%%%%%%%%%%%%%%%%%%%%%%%%%%%%%%%%%%%%%%%%%%%%%%%%%%%%%%%%%%%%%%%%%
% 在这里 include 所有参考代码
% %%%%%%%%%%%%%%%%%%%%%%%%%%%%%%%%%%%%%%%%%%%%%%%%%%%%%%%%%%%%%%%%%%%%%%%%%%%%%
% 如果它是 paper
\if 0\isBeamerMode\relax
\fi



\end{document}
