% %%%%%%%%%%%%%%%%%%%%%%%%%%%%%%%%%%%%%%%%%%%%%%%%%%%%%%%%%%%%%%%%%%%%%%%%%%%%%
% 在这里填入题目
% %%%%%%%%%%%%%%%%%%%%%%%%%%%%%%%%%%%%%%%%%%%%%%%%%%%%%%%%%%%%%%%%%%%%%%%%%%%%%
\def\sectionName{题目题目题目}



% 如果它是 beamer
\if 1\isBeamerMode\relax
    \section[\TOCName]{\sectionName}
\fi
% 如果它是 paper
\if 0\isBeamerMode\relax
    \section[\TOCName\ -\ \sectionName]{\sectionName}
\fi

\begin{frame}

% 如果它是 beamer
\if 1\isBeamerMode\relax
    {\Huge \sectionName}\par
\fi

% %%%%%%%%%%%%%%%%%%%%%%%%%%%%%%%%%%%%%%%%%%%%%%%%%%%%%%%%%%%%%%%%%%%%%%%%%%%%%
% 在这里填入你的名字
% %%%%%%%%%%%%%%%%%%%%%%%%%%%%%%%%%%%%%%%%%%%%%%%%%%%%%%%%%%%%%%%%%%%%%%%%%%%%%
\sectionAuthor{出题人出题人出题人}{作者作者作者}



% %%%%%%%%%%%%%%%%%%%%%%%%%%%%%%%%%%%%%%%%%%%%%%%%%%%%%%%%%%%%%%%%%%%%%%%%%%%%%
% 这里可以写感想(嘲讽,bushi),也可以不写!!!
% %%%%%%%%%%%%%%%%%%%%%%%%%%%%%%%%%%%%%%%%%%%%%%%%%%%%%%%%%%%%%%%%%%%%%%%%%%%%%
感想感想感想。



\end{frame}

% %%%%%%%%%%%%%%%%%%%%%%%%%%%%%%%%%%%%%%%%%%%%%%%%%%%%%%%%%%%%%%%%%%%%%%%%%%%%%
% 这里开始写简单的题目意思 ~
% %%%%%%%%%%%%%%%%%%%%%%%%%%%%%%%%%%%%%%%%%%%%%%%%%%%%%%%%%%%%%%%%%%%%%%%%%%%%%
\subsection{题目意思}
\begin{frame} % 如果一个 frame 写不下的话,多开几个就好了~
题目意思题目意思题目意思。
\end{frame}



% %%%%%%%%%%%%%%%%%%%%%%%%%%%%%%%%%%%%%%%%%%%%%%%%%%%%%%%%%%%%%%%%%%%%%%%%%%%%%
% 这里开始写题解 ~
% %%%%%%%%%%%%%%%%%%%%%%%%%%%%%%%%%%%%%%%%%%%%%%%%%%%%%%%%%%%%%%%%%%%%%%%%%%%%%
\subsection{题解}
\begin{frame}
题解题解题解。
\end{frame}


\begin{frame}[fragile] % 如果一个 frame 写不下的话,多开几个就好了~
题解题解题解。

一般而言只有代码需要使用到定宽的样式。比如 \cmd{fpow(a, b)},其他情况下用普通的
样式就可以了。Less is more。I do not like C++。

建议全部用全宽标点,中文和西文字符之间应该有一个空格。

% 如果要添加代码,那么 \begin{frame} 后面需要添加 [fragile] 选项。
% 代码片段不建议超过 10 行。
\begin{lstlisting}[language=C++]
#include <bits/stdc++.h>
using namespace std;

int main(){
    printf("Hello World");
    return 0;
}
\end{lstlisting}
\end{frame}
