
% %%%%%%%%%%%%%%%%%%%%%%%%%%%%%%%%%%%%%%%%%%%%%%%%%%%%%%%%%%%%%%%%%%%%%%%%%%%%%
% 在这里填入题目
% %%%%%%%%%%%%%%%%%%%%%%%%%%%%%%%%%%%%%%%%%%%%%%%%%%%%%%%%%%%%%%%%%%%%%%%%%%%%%
\def\sectionName{字符串 - 基础字符串知识}



% 如果它是 beamer
\if 1\isBeamerMode\relax
    \section[\TOCName]{\sectionName}
\fi
% 如果它是 paper
\if 0\isBeamerMode\relax
    \section[\TOCName\ -\ \sectionName]{\sectionName}
\fi

\begin{frame}

% 如果它是 beamer
\if 1\isBeamerMode\relax
    {\Huge \sectionName}\par
\fi

% %%%%%%%%%%%%%%%%%%%%%%%%%%%%%%%%%%%%%%%%%%%%%%%%%%%%%%%%%%%%%%%%%%%%%%%%%%%%%
% 在这里填入你的名字
% %%%%%%%%%%%%%%%%%%%%%%%%%%%%%%%%%%%%%%%%%%%%%%%%%%%%%%%%%%%%%%%%%%%%%%%%%%%%%
\sectionAuthor{Ding Yuyang}



% %%%%%%%%%%%%%%%%%%%%%%%%%%%%%%%%%%%%%%%%%%%%%%%%%%%%%%%%%%%%%%%%%%%%%%%%%%%%%
% 这里可以写感想(嘲讽,bushi),也可以不写!!!
% %%%%%%%%%%%%%%%%%%%%%%%%%%%%%%%%%%%%%%%%%%%%%%%%%%%%%%%%%%%%%%%%%%%%%%%%%%%%%



\end{frame}

% %%%%%%%%%%%%%%%%%%%%%%%%%%%%%%%%%%%%%%%%%%%%%%%%%%%%%%%%%%%%%%%%%%%%%%%%%%%%%
% 这里开始写简单的题目意思 ~
% %%%%%%%%%%%%%%%%%%%%%%%%%%%%%%%%%%%%%%%%%%%%%%%%%%%%%%%%%%%%%%%%%%%%%%%%%%%%%
\subsection{C/C++ 字符串基本知识}
\begin{frame}{char 类型} % 如果一个 frame 写不下的话,多开几个就好了~
\begin{itemize}
    \item \texttt{char ch='a';} 实际上存储的是字符的 ascii 码
    \item 'a' + 3 注意单引号和双引号不同
\end{itemize}
\end{frame}

\begin{frame}{C 风格字符串}
\begin{itemize}
    \item 字符数组
    \item \texttt{char s[6] = "abcde"}
\end{itemize}
\end{frame}

\begin{frame}{字符串结束符}
\begin{itemize}
    \item 末尾添加 '\textbackslash 0' 表示字符串结束
    \item \texttt{char s[6] = "Hello"};
    \item \texttt{char s[6] = \{'H', 'e', 'l', 'l', 'o', '\textbackslash 0'\}};
    \item \texttt{char s[6] = \{'H', 'e', 'l', 'l', 'o'\}};
    \item \texttt{char s[] = "Hello";}
\end{itemize}
\end{frame}

\begin{frame}{字符串读入}
\begin{itemize}
    \item \texttt{s[0]} 是首个字符
    \item \texttt{scanf("\%s", s);}
    \item \texttt{scanf("\%s", \&s[0]);}
    \item \texttt{cin >> s;}
\end{itemize}

\begin{itemize}
    \item \texttt{s[1]} 是首个字符
    \item \texttt{scanf("\%s", s + 1);}
    \item \texttt{scanf("\%s", \&s[1]);}
    \item \texttt{cin >> s + 1;}
\end{itemize}
\end{frame}

\begin{frame}{整行读入}
\begin{itemize}
    \item \texttt{gets(str);}
    \item \texttt{getline(cin, str); // str type = string}
    \item \texttt{fgets(str, N, stdin);}
\end{itemize}
\end{frame}

\begin{frame}{字符串函数}
\begin{itemize}
    \item \texttt{strlen(s);}
    \item 返回从s开始的字符串长度(不包括'\textbackslash 0')
    \item 复杂度为 $O(|S|)$
    \item \texttt{strlen(s + 1)}
\end{itemize}
\end{frame}

\begin{frame}[fragile]{一个常见的错误}
\begin{lstlisting}[language=C++]
for (int i = 0; i < strlen(s); i++) {
    // ......
}
\end{lstlisting}
\end{frame}

\begin{frame}{字符串函数}
\begin{itemize}
    \item \texttt{strcpy(str1, str2);}
    \item \texttt{strcmp(str1, str2);}
\end{itemize}
\end{frame}

\begin{frame}{string 读入}
\begin{itemize}
    \item \texttt{cin >> s;}
    \item \texttt{cout << s;}
    \item \texttt{s.size();} 时间复杂度: $O(1)$
\end{itemize}
\end{frame}

% %%%%%%%%%%%%%%%%%%%%%%%%%%%%%%%%%%%%%%%%%%%%%%%%%%%%%%%%%%%%%%%%%%%%%%%%%%%%%
% 这里开始写题解 ~
% %%%%%%%%%%%%%%%%%%%%%%%%%%%%%%%%%%%%%%%%%%%%%%%%%%%%%%%%%%%%%%%%%%%%%%%%%%%%%
% \subsection{题解}
% \begin{frame}
% 题解题解题解。
% \end{frame}
% 
% 
% \begin{frame}[fragile] % 如果一个 frame 写不下的话,多开几个就好了~
% 题解题解题解。
% 
% 一般而言只有代码需要使用到定宽的样式。比如 \cmd{fpow(a, b)},其他情况下用普通的
% 样式就可以了。Less is more。I do not like C++。
% 
% 建议全部用全宽标点,中文和西文字符之间应该有一个空格。
% 
% % 如果要添加代码,那么 \begin{frame} 后面需要添加 [fragile] 选项。
% % 代码片段不建议超过 10 行。
% \begin{lstlisting}[language=C++]
% #include <bits/stdc++.h>
% using namespace std;
% 
% int main(){
%     printf("Hello World");
%     return 0;
% }
% \end{lstlisting}
% \end{frame}
