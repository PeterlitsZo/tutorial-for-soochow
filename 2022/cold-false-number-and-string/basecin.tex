
% %%%%%%%%%%%%%%%%%%%%%%%%%%%%%%%%%%%%%%%%%%%%%%%%%%%%%%%%%%%%%%%%%%%%%%%%%%%%%
% 在这里填入题目
% %%%%%%%%%%%%%%%%%%%%%%%%%%%%%%%%%%%%%%%%%%%%%%%%%%%%%%%%%%%%%%%%%%%%%%%%%%%%%
\def\sectionName{字符串 - 基础字符串知识}



% 如果它是 beamer
\if 1\isBeamerMode\relax
    \section[\TOCName]{\sectionName}
\fi
% 如果它是 paper
\if 0\isBeamerMode\relax
    \section[\TOCName\ -\ \sectionName]{\sectionName}
\fi

\begin{frame}

% 如果它是 beamer
\if 1\isBeamerMode\relax
    \noindent {\Huge \sectionName}\par
\fi

% %%%%%%%%%%%%%%%%%%%%%%%%%%%%%%%%%%%%%%%%%%%%%%%%%%%%%%%%%%%%%%%%%%%%%%%%%%%%%
% 在这里填入你的名字
% %%%%%%%%%%%%%%%%%%%%%%%%%%%%%%%%%%%%%%%%%%%%%%%%%%%%%%%%%%%%%%%%%%%%%%%%%%%%%
\sectionAuthor{Ding Yuyang}



% %%%%%%%%%%%%%%%%%%%%%%%%%%%%%%%%%%%%%%%%%%%%%%%%%%%%%%%%%%%%%%%%%%%%%%%%%%%%%
% 这里可以写感想(嘲讽,bushi),也可以不写!!!
% %%%%%%%%%%%%%%%%%%%%%%%%%%%%%%%%%%%%%%%%%%%%%%%%%%%%%%%%%%%%%%%%%%%%%%%%%%%%%



\end{frame}

% %%%%%%%%%%%%%%%%%%%%%%%%%%%%%%%%%%%%%%%%%%%%%%%%%%%%%%%%%%%%%%%%%%%%%%%%%%%%%
% 这里开始写简单的题目意思 ~
% %%%%%%%%%%%%%%%%%%%%%%%%%%%%%%%%%%%%%%%%%%%%%%%%%%%%%%%%%%%%%%%%%%%%%%%%%%%%%
\subsection{C/C++ 字符串基本知识}
\begin{frame}{char 类型} % 如果一个 frame 写不下的话,多开几个就好了~
\begin{itemize}
    \item \cmd{char ch='a';} 实际上存储的是字符的 ascii 码。
    \item \cmd{'a' + 3} 注意单引号和双引号不同。
\end{itemize}
\end{frame}

\begin{frame}{C 风格字符串}
\begin{itemize}
    \item 字符数组。
    \item \cmd{char s[6] = "abcde";}
\end{itemize}
\end{frame}

\begin{frame}{字符串结束符}
\begin{itemize}
    \item 末尾添加 \cmd{'\textbackslash 0'} 表示字符串结束。
    \item \texttt{char s[6] = "Hello";}
    \item \texttt{char s[6] = \{'H', 'e', 'l', 'l', 'o', '\textbackslash 0'\};}
    \item \texttt{char s[6] = \{'H', 'e', 'l', 'l', 'o'\};}
    \item \texttt{char s[] = "Hello";}
\end{itemize}
\end{frame}

\begin{frame}{字符串读入}
\begin{itemize}
    \item \texttt{s[0]} 是首个字符。
    \item \texttt{scanf("\%s", s)}。
    \item \texttt{scanf("\%s", \&s[0])}。
    \item \texttt{cin >> s}。
\end{itemize}

\begin{itemize}
    \item \texttt{s[1]} 是首个字符。
    \item \texttt{scanf("\%s", s + 1)}。
    \item \texttt{scanf("\%s", \&s[1])}。
    \item \texttt{cin >> s + 1}。
\end{itemize}
\end{frame}

\begin{frame}{整行读入}
\begin{itemize}
    \item \texttt{gets(str)}。
    \item \texttt{getline(cin, str)},这里 \cmd{str} 的类型应该是 \cmd{string}。
    \item \texttt{fgets(str, N, stdin)}。
\end{itemize}
\end{frame}

\begin{frame}{字符串函数}
\begin{itemize}
    \item \texttt{strlen(s)}。
    \item 返回从 \cmd{s} 下标 $0$ 开始的字符串长度(不包括 \cmd{'\textbackslash
        0'})。
    \item 复杂度为 $O(|S|)$。
    \item 如果下标从 $1$ 开始的话,则使用 \texttt{strlen(s + 1)}。
\end{itemize}
\end{frame}

\begin{frame}[fragile]{一个常见的错误}
\begin{lstlisting}[language=C++]
for (int i = 0; i < strlen(s); i++) {
    // ......
}
\end{lstlisting}

时间复杂度不是 $O(|S|)$ 而是 $O(|S|^2)$。
\end{frame}

\begin{frame}{字符串函数}
\begin{itemize}
    \item \texttt{strcpy(str1, str2)},用于复制。
    \item \texttt{strcmp(str1, str2)},用于比较。
\end{itemize}
\end{frame}

\begin{frame}{string 读入}
\begin{itemize}
    \item \texttt{cin >> s},字符串输入。
    \item \texttt{cout << s},字符串输出。
    \item \texttt{s.size()},时间复杂度: $O(1)$。
\end{itemize}
\end{frame}
