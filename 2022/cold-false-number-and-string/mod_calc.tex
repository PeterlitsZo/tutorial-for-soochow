
% %%%%%%%%%%%%%%%%%%%%%%%%%%%%%%%%%%%%%%%%%%%%%%%%%%%%%%%%%%%%%%%%%%%%%%%%%%%%%
% 在这里填入题目
% %%%%%%%%%%%%%%%%%%%%%%%%%%%%%%%%%%%%%%%%%%%%%%%%%%%%%%%%%%%%%%%%%%%%%%%%%%%%%
\def\sectionName{数论 - 模运算}



% 如果它是 beamer
\if 1\isBeamerMode\relax
    \section[\TOCName]{\sectionName}
\fi
% 如果它是 paper
\if 0\isBeamerMode\relax
    \section[\TOCName\ -\ \sectionName]{\sectionName}
\fi

\begin{frame}

% 如果它是 beamer
\if 1\isBeamerMode\relax
    \noindent{\Huge \sectionName}\par
\fi

% %%%%%%%%%%%%%%%%%%%%%%%%%%%%%%%%%%%%%%%%%%%%%%%%%%%%%%%%%%%%%%%%%%%%%%%%%%%%%
% 在这里填入你的名字
% %%%%%%%%%%%%%%%%%%%%%%%%%%%%%%%%%%%%%%%%%%%%%%%%%%%%%%%%%%%%%%%%%%%%%%%%%%%%%
\sectionAuthor{Peterlits Zo}



% %%%%%%%%%%%%%%%%%%%%%%%%%%%%%%%%%%%%%%%%%%%%%%%%%%%%%%%%%%%%%%%%%%%%%%%%%%%%%
% 这里可以写感想(嘲讽,bushi),也可以不写!!!
% %%%%%%%%%%%%%%%%%%%%%%%%%%%%%%%%%%%%%%%%%%%%%%%%%%%%%%%%%%%%%%%%%%%%%%%%%%%%%
\noindent 模运算,模来模去!



\end{frame}

% %%%%%%%%%%%%%%%%%%%%%%%%%%%%%%%%%%%%%%%%%%%%%%%%%%%%%%%%%%%%%%%%%%%%%%%%%%%%%
% 取模
% %%%%%%%%%%%%%%%%%%%%%%%%%%%%%%%%%%%%%%%%%%%%%%%%%%%%%%%%%%%%%%%%%%%%%%%%%%%%%
\subsection{取模}

\begin{frame}
首先我们来定义一哈取模:对于数 $a$ 和 $b$ 而言,我们找到 $a$ 除以 $b$ 之后的余数,
那就是取模啦!真简单。其中,我们把运算符定义为 $\bmod$。\pause

举例子来说明:
\begin{itemize}
    \item $5 \bmod 4 = 1$。\pause
    \item $100 \bmod 5 = 0$。\pause
    \item $5 \bmod 3 = 2$。
\end{itemize}

在 C++ 下,我们用符号 \cmd{\%} 来表示取余。
\end{frame}

\begin{frame}
我们定义完了取余,接下来我们就定义一哈同余。假设模数是 $m$(mod 的首字母~),而
有两个数 $a$ 和 $b$,如果满足:\[
    a \bmod m = b \bmod m
\]\pause
那么我们就说在模 $m$ 下,$a$ 和 $b$ \textbf{同余}。

我们记为:\[
    a \equiv b \pmod m
\]
\end{frame}

\begin{frame}
可以注意到一个很小的性质,那就是,对于任意一个数 $n$ 而言,都存在一个整数 $i \in
[0, m - 1]$,使得 $n$ 和 $i$ 同余。

还有一个性质,那就是 $a$ 和 $b$ 同余,当且仅当 $a - b \equiv 0 \pmod m$。
\end{frame}

\subsection{模运算的性质}

\begin{frame}
我们现在需要深入研究一哈模运算的性质。

首先是:\[
    a + b \equiv (a \bmod m) + (b \bmod m) \pmod m
\] \pause

为了证明这个,我们不妨假设 $a' = a \bmod m$,同理有 $b'$。那么 $a = a' + p_a m$,
同理 $b = b' + p_b m$。

那么代入有:\[
    a' + p_a m + b' + p_b m = a' + b' + (p_a + p_b) m
\]

那么左边减右边,得到 $(p_a + p_b) m$,这个很明显在模 $m$ 下和 $0$ 同余嘛。
\end{frame}

\begin{frame}[fragile]
同理对于减法也是这样的。不过说道减法,一般而言,C++ 的 \cmd{\%} 取模运算符并不能
让负数也能得到在 $[0, m - 1]$ 部分的模。这是因为在 C++ 中有:
\begin{lstlisting}[language=C++]
(-a) % b = -(a % b)
\end{lstlisting}\pause
所以说,一般而言,我们会使用下面的东西来得到对应的值:
\begin{lstlisting}[language=C++]
(a % b + b) % b
\end{lstlisting}
\end{frame}

\begin{frame}
同理我们有:\[
    a \times b \equiv (a \bmod m) \times (b \bmod m) \pmod m
\] \pause

证明方法和证明加法很像。这里就略去证明了。
\end{frame}

\begin{frame}
那么除法呢?这个不太好讲,我们可以留到后面讲逆元的时候再讲。但是现在我们需要知道
的是,对于除法而言,下面的式子是不成立的:\[
    a / b \equiv (a \bmod m) / (b \bmod m) \pmod m
\]

比如说,$a = 6$,$b = 3$,而 $m = 5$。更何况,我们难过地发现,除法带来了小数。而
后面我们可以通过学逆元惊喜地发现,我们可以让除法搞不出小数来。
\end{frame}
